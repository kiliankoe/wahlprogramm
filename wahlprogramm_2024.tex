% das Papierformat zuerst
\documentclass[a4paper, 11pt]{article}
%\pagestyle{headings}
\usepackage{graphicx}
\usepackage[ngerman]{babel} % deutsche schriftzeichen
\usepackage[T1]{fontenc}    % deutsche schriftzeichen, nutze 'fontspec' bei lualatex
%\usepackage[utf8]{inputenc} % deutsche schriftzeichen -msw; geht irgendwie nicht beim hauselaptop
\usepackage[latin9]{inputenc}
\usepackage{lmodern} % latin modern schriftart, soll angeblich im pdf besser lesbar sein
\usepackage{layout}
\usepackage{array}

%\usepackage{showframe}

\setlength{\parindent}{0pt}
\setlength{\textwidth}{16cm}
\setlength{\oddsidemargin}{0cm}
\setlength{\evensidemargin}{0cm}
\setlength{\hoffset}{0cm}

%\renewcommand{\contentsname}{Table of things}
\addto\captionsngerman{%
  \renewcommand{\figurename}{Figaaa.}%
  \renewcommand{\contentsname}{Inhalt}%
}


% hier beginnt das Dokument
\begin{document}
%\layout{}
\title{\Huge{\bf{Wahlprogramm zur \\ Kommunalwahl 2024}}}

\author{\Huge{PIRATEN Dresden}}
\date{\Large{2024}}
\maketitle
\thispagestyle{empty}

\vspace*{8cm}\hspace*{3cm}\includegraphics[width=10cm]{P2D2.eps}%\par\vspace{6cm}
%\vspace*{8cm}\hspace*{3cm}\includegraphics[width=10cm]{P2D2.pdf}%\par\vspace{6cm}
%\vspace{-1.8cm}
%\mbox{Dresden, \today} \newline
%\mbox{Dr.~Martin Schulte-Wissermann, PIRATEN}\newline
\newpage

%seite 2 ohne nummer

\thispagestyle{empty}
.

% dritte seite geht mit 1 los
\newpage
\clearpage
\pagenumbering{arabic}
\tableofcontents

\newpage

\section{Präambel}
Jeder Mensch ist mit gleichen Rechten geboren. Niemand darf daher aufgrund von Merkmalen jeglicher Art benachteiligt oder gar ausgeschlossen werden. Piratige Politik muss sich immer von diesem Grundsatz leiten lassen. Insbesondere Kinder und Jugendliche stehen unter besonderem Schutz, denn sie konnten sich ihr Umfeld nicht selbst aussuchen. Daher ist es ein oberes Piratenziel, gleiche Teilhabe am städtischen Leben für alle Menschen und vor allem für Kinder und Jugendliche zu ermöglichen.\newline
\vspace*{0.25cm}

Wir Piraten sehen, dass wir in einer unglaublich ungerechten Welt leben und müssen das ändern. Insbesondere muss die Grundversorgung aller essenziellen Bedarfe des Menschen generell gesichert sein. Wir sehen gesellschaftliche, kulturelle und politische Teilhabe, Bildung, Gesundheit, Transport und eine ausreichende monetäre Absicherung als essenziell an.\newline
\vspace*{0.25cm}

Die Zukunft bietet Chancen, die es zu nutzen und positiv zu gestalten gilt. Diese Gestaltung der Zukunft begreifen wir als einen gesellschaftlichen Prozess. Daraus leitet sich ab, dass jeder Mensch in der Gesellschaft eingeladen sein muss, sich einzubringen und die Zukunft mitzugestalten. Zukunft ist immer ein dauerhafter Prozess von allen Individuen. Hierzu muss aber auch die Stadt den Menschen die Möglichkeit bieten, sich aktiv an der Zukunft zu beteiligen. Die Informationen müssen frei und transparent zugänglich sein, der Mensch muss gehört und ernst genommen werden – und schließlich muss die Stadtpolitik auch Teile ihrer Entscheidungsmacht abgeben.\newline
\vspace*{0.25cm}

Neue Technologien bieten Chancen, unsere Welt besser zu machen. Wir Piraten lieben neue Technologien und wollen ihre positiven Möglichkeiten für Bildung, Information und Kooperation überall fördern. Neue Chancen stellen aber immer auch eine Gefahr dar, die Welt unfreier, ungerechter und unsicherer zu machen. Wir Piraten haben keine Angst – im Gegenteil, wir stellen uns allen Fehlentwicklungen von Überwachung, Einschüchterung und Kontrolle entschlossen entgegen.\newline
\vspace*{0.25cm}

Wir Piraten glauben an die individuelle Freiheit der Menschen. Daraus folgt, dass Verbote und Restriktionen, wo immer möglich, zu vermeiden und durch Angebote zu ersetzen sind. Wir alle müssen miteinander leben, das erfordert Respekt und Verantwortung. Was keinem anderen schadet, darf nicht verboten sein. Was verboten ist, darf dadurch keine Schäden verursachen.\newline
\vspace*{0.25cm}

Wir Piraten glauben, dass die Menschen unglaublich schlau sind und neue Ideen entwickeln können – auch und gerade zur besseren Gestaltung einer sich verdichtenden Stadt. Man muss nur über den Tellerrand schauen, bei anderen lernen und neue Dinge ausprobieren. Dies ist ein spannender Prozess, und wir wollen gern neue Ideen aufgreifen und gemeinsam umsetzen.\newline
%\vspace*{0.4cm}
\newpage

Bei Energie und Verkehr ist die Welt in einer Sackgasse gelandet. Die Städte dieser Welt haben eine besondere Verantwortung, uns mit neuen Ideen, regenerativen Energien und besseren Verkehrskonzepten die Zukunft zu sichern. Die Piraten sind überzeugt, dass Dresden hier eine Vorreiterrolle übernehmen kann. Wir müssen nur die falschen Dinge lassen und die richtigen Dinge tun.\newline
%\vspace*{0.2cm}

Dresden ist selbstbewusst – jeder Mensch in Dresden macht die Stadt und ihren Reiz aus. Wir treten in diesem Kommunalwahlkampf an, um gemeinsam etwas zu verändern, um Verkrustungen aufzubrechen und weiterhin frischen Wind in die Politik Dresdens zu bringen.\newline

Wir sind motiviert. Wir haben Ideen. Wir sind Piraten.

\section{Grundsätzliches}
\subsection{Integration des Wahlprogramms der Neustadtpiraten}
Die PIRATEN Dresden sehen die im Wahlprogramm der Neustadtpiraten formulierten Ziele als Teil des Dresdner Wahlprogramms an.

\subsection{Piraten helfen!}
Politischen und gesellschaftlichen Initiativen, deren Werte und Ziele wir teilen, helfen wir gerne.

% ###################
\section{Netzpolitik}

\subsection{Medienkompetenz stärken}
Die PIRATEN Dresden setzen sich für die Förderung der Medienkompetenz ein. Diese muss in allen Altersstufen vermittelt werden, angefangen von Kindergarten und Schule bis zu Seniorenkursen. Jeder Mensch muss wissen, wie man Falschnachrichten erkennt, was eine Filterblase ist, welche Daten man bei welchen Diensten preisgibt, wie man sicher und anonym surft sowie miteinander kommuniziert.


\subsection{Open-Data Initiative}
Die PIRATEN Dresden fordern die datenschutzkonforme Bereitstellung aller öffentlichen Daten in maschinenlesbarem Format.

\subsection{Freifunk}
Die PIRATEN Dresden setzen sich für die Förderung von Freifunk-Initiativen ein. Dies kann durch die Unterstützung der Softwareentwicklung oder durch Bereitstellung von öffentlichen Gebäuden und Netzinfrastruktur geschehen.


\subsection{WLAN im Park}
Die PIRATEN Dresden setzen sich dafür ein, dass Dresdener Parks mit WLAN versorgt werden. Hierzu sind Freifunk-Initiativen zu unterstützen. Ein kommunaler Beitrag ist hierbei sehr wünschenswert.

\subsection{Open-Source in der Verwaltung}
Die PIRATEN Dresden setzen sich für die vollständige Umstellung von proprietärer Software hin zu freier Open-Source-Software (FLOSS) in der Stadtverwaltung und den städtischen Betrieben ein.

\subsection{Public Money Public Code}
Jegliche für die Verwaltung erstellte Software muss unter einer Open-Source-Lizenz zur Verfügung gestellt werden.

\subsection{Personenbezogene Daten schützen}
Die PIRATEN Dresden fordern, dass personenbezogene Daten (z. B. Meldedaten) nicht gegen den eigenen Willen an Dritte weitergegeben und schon gar nicht verkauft werden dürfen (Opt-in oder Opt-out).

\subsection{eGovernment $-$ elektronische Verwaltung}
Die PIRATEN Dresden wollen die Möglichkeiten des Internets auch für eine verbesserte Zusammenarbeit zwischen Menschen/Institutionen und der Verwaltung einsetzen (eGovernment). Dies muss auf einer sicheren, kostenlosen, verschlüsselten und barrierearmen IT beruhen. Innerhalb der Verwaltung sollen die Möglichkeiten der Digitalisierung genutzt werden, um Vorgänge für Bürger·innen schneller abzuwickeln und die Arbeitslast für die Mitarbeitenden zu reduzieren. Möglichst sollen alle Anliegen vollständig online abgewickelt werden können. Das soll für Angehörige aller Staaten, ob EU oder nicht-EU, möglich sein. Der Postweg ist als Alternative zwingend zu erhalten.

\subsection{Sicherer E-Mailverkehr zwischen Stadt und Einwohnenden}
Jede Kommunikation der Dresdner Verwaltung, sowohl rein interne als auch externe, muss im Normalfall verschlüsselt sein. Diese Verschlüsselung muss einschlägige Open-Source-Standards (bspw. S/MIME) verwenden und für Bürger·innen kostenfrei sein. Die erforderliche Infrastruktur (Zertifikats-Vergabe-Verfahren, Verzeichnisdienst für öffentliche Schlüssel) soll sich ebenfalls ausschließlich an Open-Source-Software bedienen. Die Vorgaben für Zugänge, Ausgabe und weitere Verwendungen der kryptografischen Schlüssel müssen öffentlich im Internet einsehbar sein.


% ###################
\section{Freiheit und Selbstbestimmung}

\subsection{Versammlungsrecht stärken}
Die PIRATEN Dresden wenden sich entschieden gegen bestehende und geplante Einschränkungen des Versammlungsrechts. Das Demonstrationsrecht ist ein Grundrecht, welches nicht durch polizeiliche und politische Willkür eingeschränkt werden darf. Demonstrationen sind grundsätzlich in Sicht- und Hörweite zuzulassen.


\subsection{Videoüberwachung in Dresden abbauen}
Die PIRATEN Dresden lehnen Videoüberwachung generell ab und fordern den Abbau der bestehenden Videoüberwachung im öffentlichen Raum. Vor allem wenden wir uns entschieden gegen die neue Generation an intelligenten und vernetzten Kameras, welche das Niveau der Überwachung und Kontrolle durch Gesichtserkennung und Personenidentifikation auf ein Orwellsches Level hebt.




\subsection{Versammlungsbehörde demokratisieren}
Die PIRATEN Dresden fordern, dass die Versammlungsbehörde der Stadt entgegen der bisheriger Praxis davon ablassen muss, demokratische Demonstrationen einseitig zu beschneiden. Sie muss zu einer Behörde werden, die demokratischen Protest ermöglicht, statt ihn zu schikanieren.


\subsection{Polizeigesetze entschärfen}
Die PIRATEN Dresden setzen sich auf allen politischen und gesellschaftlichen Ebenen dafür ein, dass das Sächsische Polizeigesetz wieder entschärft wird. Unter keinen Umständen darf es eine weitere Verschärfung des Polizeigesetzes geben. Insbesondere die Einführung einer Präventivhaft lehnen wir entschieden ab. Sollte eine Verschärfung des Polizeigesetzes im Landtag beschlossen, werden die PIRATEN Dresden alle kommunalpolitischen Mittel ausschöpfen, um die Auswirkungen abzumildern, die Menschen über ihre Rechte zu informieren und den gesellschaftlich-politischen Protest gegen den ``Überwachungs- und Kontrollwahn'' des Staates auf die Straße zu bringen.


\subsection{Freiraum Assi-Eck}
Die PIRATEN Dresden fordern, dass das Recht sich im öffentlichen Raum aufzuhalten in keinster Weise eingeschränkt wird. Dies gilt insbesondere auch für sogenannte 'Hotspots' $-$ wir Menschen haben immer und überall das Recht, uns an solchen Orten aufzuhalten und unsere Freizeit dort zu verbringen. Gegenseitige Rücksicht gebietet, dass es dabei nicht zu Konflikten kommt. Hierbei können 'Nachtschlichter·innen' ein befriedendes Element darstellen. Speziell für das Assi-Eck fordern die PIRATEN Dresden, dass (zumindest am Wochenende) die Kreuzung Louisenstraße-Rothenburger-/Görlitzer Straße weiträumig autofrei gehalten wird, um die heute bestehenden schweren Konflikte zwischen Autos und Menschen zu vermeiden. Den Einsatz von Polizei und Ordnungsamt gegen friedliche Menschen lehnen wir entschieden ab.


\subsection{Ordnungsamt entwaffnen}
Die PIRATEN Dresden fordern, dass das Ordnungsamt optisch wie funktional wieder auf seine originären Aufgaben ausgerichtet wird. Polizeiliche und polizeiähnliche Tätigkeiten hat das Ordnungsamt nicht auszuüben. Polizeiähnliche Uniformen, Bewaffnung und das Auftreten als Sicherheitsorgan sind nicht Bestandteil ordnungsamtlicher Arbeit.


\subsection{Cannabis Social Clubs}
Wir setzen uns dafür ein, dass in Dresden Cannabis Social Clubs eingerichtet werden.


\subsection{Drogenprävention und sicherer Drogenkonsum}
Die PIRATEN Dresden setzen sich dafür ein, dass die Drogenprävention und die Suchthilfe finanziell, personell und strukturell gestärkt wird. Darüber hinaus fordern wir, dass anonyme Teststationen geschaffen werden, in denen man die genaue Zusammensetzung von Substanzen bestimmen lassen kann.


\subsection{Freiräume}
Die PIRATEN Dresden setzen sich dafür ein, Leerstand und ungenutzte Flächen in der Stadt zu katalogisieren und Kunstschaffenden und Kulturinitiativen zur Verfügung zu stellen. Wir unterstützen Initiativen, die ähnliche Ziele verfolgen. Kommunale Flächen, die für eine zukünftige Nutzung vorgesehen sind, sollen als ``Freiraum auf Zeit'' zur Zwischennutzung bereitgestellt werden.


\subsection{Erklärung zur BRN}
Die Neustadtpiraten und die PIRATEN Dresden bekennen sich zur BRN und wissen um die besondere Bedeutung dieses Stadtteilfestes für die Äußere Neustadt. Wir legen Wert auf Zusammenarbeit mit den örtlichen Bürger*innen-Bewegungen (z. B. der Schwafelrunde bzw. deren Nachfolger·innen) und verfolgen das gemeinsame Bemühen um eine positive Entwicklung unseres Stadtteilfestes. Aufgrund der steigenden und im Ehrenamt kaum mehr stemmbaren bürokratischen Belastung ist die Zukunft der BRN noch offen. Wir sind gegen eine fortschreitende Kommerzialisierung und für mehr Beteiligung der Bewohnenden der Dresdner Neustadt bei der Frage, wie die BRN in Zukunft organisiert werden soll. Das Fest gehört den Neustädter·innen in ihrer Tradition der unabhängigen Bunten Republik Neustadt $-$ und deshalb können auch nur sie über ihre Zukunft entscheiden.



\subsection{Prävention statt Repression}
Die PIRATEN Dresden fordern den Stopp von willkürlichen Personenkontrollen und Belästigungen, insbesondere Jugendlicher und durch Racial Profiling, seitens der Polizei und des Ordnungsamts. Ein solches Vorgehen erzeugt nur Ablehnung und Unmut gegenüber staatlichen Institutionen. Daher sind solche repressiven Ansätze im besten Fall ``symbolische pseudo-Symptombehandlung'' $-$ im schlechtesten Fall treiben sie insbesondere Jugendliche direkt in die Arme der Kriminalität. Vielmehr müssen mehr Sozialarbeitende sowie Angebote der Jugendhilfe und Beratungsstellen die Sorgen und Bedürfnisse der Betroffenen aufnehmen und positiv kanalisieren.


\subsection{Förderung von nicht-kommerziellen Räumen für Kreativität und Wissensvermittlung}
In Dresden gibt es viele nicht-kommerzielle Vereine und Initiativen, die Raum für Kreativität, Innovation und Wissensvermittlung bieten, sowie das aktive Gestalten der eigenen Welt fördern. Wir PIRATEN Dresden fordern, dass solche Vereine und Initiativen, wie bspw. Hackspaces oder offene Werkstätten von der Stadt aktiv gefördert werden.


\subsection{Freitanz legalisieren}
Die PIRATEN Dresden setzen sich dafür ein, dass nicht-kommerzielle Freitanz-Veranstaltungen nach Vorbild des Bremer-Modelles legalisiert werden.



% ###########################################
\section{Bürgerpartizipation und Transparenz}

\subsection{Übertragung der Ratssitzungen}
Die PIRATEN Dresden fordern die Live-Übertragungen (Stream) und die Aufzeichnung der Stadtratssitzungen zwingend zu erhalten und stetig zu verbessern, z. B. durch Barrierefreiheit, Untertitel, CC0-Lizenz und Links zu einzelnen Redebeiträgen. Des Weiteren fordern wir zumindest Audio-Übertragungen der öffentlichen Ausschusssitzungen. Jegliches Depublizieren von Sitzungsaufzeichnungen lehnen wir ab.




\subsection{Wahlalter absenken}
Als Landeshauptstadt soll Dresden im Land darauf hinwirken, dass das Wahlalter für Kommunalwahlen auf 14 oder 16 Jahren herabgesetzt wird.


\subsection{Kinder- und Jugendparlament}
Die PIRATEN Dresden setzen sich dafür ein, dass in Dresden ein Kinder- und Jugendparlament geschaffen wird.


\subsection{Transparenzsatzung}
Die PIRATEN Dresden fordern, dass sich die Landeshauptstadt Dresden als eine 'transparenzpflichtige Stelle' erklärt und eine Transparenzsatzung im weitestmöglichen Umfang aufstellt.


\subsection{Stärkung der Stadtbezirksbeiräte}
Die PIRATEN Dresden setzen sich für eine weitere Stärkung der Stadtbezirksbeiräte und deren dauerhaft gesicherte finanzielle Ausstattung ein, welche sich in Zukunft an der Höhe der Finanzierung der Ortschaftsräte orientieren muss. Stadtbezirksbeiräte sind am nähsten an den Wünschen und Interessen der Einwohner·innen eines Orts- oder Stadtteils und sollen daher auch lokale Entscheidungen treffen und Ausgaben selbstständig tätigen können. Unabdingbar bei der Stärkung der Stadtbezirksbeiräte ist der Erhalt der Direktwahl der Mitglieder. Ebenso unabdingbar ist der Erhalt oder der Ausbau des Selbstbefassungsrechts $-$ also das Recht, die eigene Tagesordnung aufzustellen $-$ und die Möglichkeit, Anträge an den Stadtrat sowie Anfragen an den Oberbürgermeister zu stellen. Die PIRATEN Dresden unterstützen daher auch Initiativen auf Landesebene zur Umwandlung der Dresdner Stadtbezirksbeiräte in echte Ortschaftsräte.


\subsection{Bürger·innenbeteiligung stärken}
Die PIRATEN Dresden fordern, dass die Möglichkeiten der aktiven Bürger·innenbeteilungung gestärkt werden. Hierzu gehört eine verbesserte Information der Verwaltung gegenüber der Öffentlichkeit über (geplante) Vorgänge und eine frühzeitige und umfassende Information über Beteiligungsformate. Die bestehende Bürgerbeteiligungssatzung [1] muss endlich mit Leben gefüllt werden. Hierzu muss die Stadt aktiv die Menschen befähigen, die notwendigen Schritte hin zu Bürger·innenentscheiden/-foren gehen zu können.\newline

Die PIRATEN Dresden fordern, dass ausreichend Mittel für Bürger·innenbeiligungsmaßnahmen im Haushalt bereitgestellt werden,


\subsection{Ausschussarbeit transparenter machen}
Die Dresdner PIRATEN unterstützen Initiativen, welche auf eine Änderung des Landesrechts hinarbeiten, um die Ausschussarbeit transparenter zu machen. Bis dahin muss das Ratsinformationssystem auch alle Informationen zu nicht-öffentlichen Tagesordnungspunkten von Ausschusssitzungen präsentieren. Dies beinhaltet auch die in den Ausschüssen gezeigten Präsentationen, falls diese nicht als vertraulich gekennzeichnet sind. Als vertraulich gekennzeichnete und damit nur den Stadtratsmitgliedern zugängliche Dokumente müssen als Quellen gekennzeichnet werden. Des Weiteren soll von den nicht-öffentlichen Sitzungen ein Ergebnis-Protokoll inkl. Abstimmungsergebnissen zeitnah veröffentlicht werden.


\subsection{Verständlicher Haushaltsplan}
Die PIRATEN Dresden fordern, dass der städtische Haushalt schon in der Entwurfsphase in maschinenlesbarer Form öffentlich zugänglich ist und zugänglich bleibt. Außerdem ist die Partizipation von Vereinen und Einzelpersonen durch geeignete, bspw. elektronische Verfahren sicherzustellen, um Kommentare, Änderungsvorschläge und Kritik öffentlich dokumentieren zu können.


\subsection{Informationsfreiheitssatzung}
Die PIRATEN Dresden fordern die Erweiterung der bestehenden Informationsfreiheitssatzung. Vor allem müssen die Gebühren entfallen.


\subsection{Whistleblowing ermöglichen und Whistleblower·innen schützen}
Whistleblower·innen (auch Hinweisgeber·innen) übernehmen in unserer Gesellschaft eine wichtige Funktion, indem sie auf Missstände aufmerksam machen und für Transparenz sorgen. Die PIRATEN Dresden fordern die Einrichtung einer von Stadt und Politik völlig unabhängigen neutralen Whistleblowing-Stelle. Die Stelle muss anonymisiert nutzbar sein.


\section{Gleichberechtigung und Gleichstellung}

\subsection{Inklusive Teilhabe in der gesamten Stadt}
Eine selbstbestimmte, vielfältige Gesellschaft gibt allen die Möglichkeit, wirklich an ihr teilzuhaben. Deshalb setzen wir uns ausdrücklich dafür ein, die Diskriminierung marginalisierter Gruppen auch mit den Mitteln der Stadt Dresden zu bekämpfen. Zum einen brauchen wir konkrete Maoderßnahmen, wie etwa die Minimierung von Barrieren beim Straßenbau und der Verkehrsführung durch abgesenkte Bordsteine, die dezentrale Unterbringung von Geflüchteten, Förderprogramme für Migration und Inklussion durch Sport- und Kultur-Projekte, die Unterstützung von demokratiebildenden Veranstaltungen sowie Pride-Events. Zum anderen sind darüber hinaus konkrete lokale Richtlinien denkbar, die den Stadtrat und die Verwaltung unterstützen, Diskriminierungsformen wie z. B.: Sexismus, Ableismus, Rassismus, Adultismus, Altersdiskriminierung oder Queerfeindlichkeit zu minimieren. Diese müssen mit Hilfe entsprechender Fachausschüsse entwickelt werden.


\subsection{Gleichberechtigung bei kommunaler Förderung konkret werden lassen}
Die PIRATEN Dresden fordern, dass jegliche kommunale Förderung an die Bedingung geknüpft ist, dass dadurch mehr Geschlechtergerechtigkeit/Gleichberechtigung erzielt wird (Umsetzung der vom Stadtrat mitunterzeichneten ``Europäischen Charta für die Gleichstellung von Frauen und Männern auf lokaler Ebene''). Unter keinen Umständen dürfen tradierte Geschlechterrollen erhalten oder gar verfestigt werden. Diese Grundsätze sind bei jeder Einzelfallentscheidung zu beachten und müssen auch in alle Förderrichtlinien der Stadt aufgenommen werden.



\subsection{Einführung des anonymen Bewerbungsverfahrens in der Stadtverwaltung und bei den städtischen Gesellschaften}
Die PIRATEN Dresden setzen sich als Ziel, das anonymisierte Bewerbungsverfahren für die Stadtverwaltung und für alle städtischen Gesellschaften und Betriebe einzuführen. Auch für die Einstellung der Auszubildenden ist dieses Verfahren anzuwenden.


\subsection{Kreuzchor auch für Mädchen}
Die PIRATEN Dresden fordern, dass nicht nur exklusiv Jungen im Kreuzchor in den Genuss einer fundierten musikalischen Ausbildung kommen. Daher muss der Kreuzchor auch einen Mädchenchor unterhalten. Beide Chöre sollen gleichwertig behandelt und finanziert werden.


\section{Umwelt und Klima}

\subsection{Insektenfreundliche Wiesen}
Die PIRATEN Dresden setzen sich für insektenfreundliche Grünflächen ein.


\subsection{Natürliche Wasserläufe und Auwald}
Die PIRATEN Dresden setzen sich dafür ein, dass kleinen und großen Gewässern wieder Platz für ihre natürliche Entwicklung gegeben wird. Dies ist wichtig sowohl für den aktiven Hochwasserschutz als auch für die Artenvielfalt. Entlang der Elbe soll der Entwicklung von Auwald Zeit und Raum gegeben werden.


\subsection{Klimaneutrale SachsenEnergie als (de-)zentrale Energiedienstleisterin}
Die Piraten Dresden fordern, dass das Dekarbonisierungskonzept der Sachsenenergie an entscheidenden Stellen nachgebessert wird:

\begin{itemize}
    \item Der späteste Zeitpunkt zum Erreichen einer vollständigen Dekarbonisierung der SachsenEnergie muss auf 2035 vorgezogen werden. Das Konzept ist entsprechend anzupassen.
    \item Ein umfassendes, jährliches Monitoring muss den Umsetzungsstand entsprechend des Planpfades aufzeigen. Bei Zeitverzögerungen sind umgehend Gegenmassnahmen zu ergreifen.
    \item Um die Einflussmöglichkeiten auf die Dekarbonisierung Dresdens als Stadt zu erhalten bzw. zu erhöhen, muss die SachsenEnergie weit stärker als geplant auf dezentrale Energieerzeugung/-speicherung setzen. Hierzu muss die SachsenEnergie auf alle Akteure (Haus-/Grundbesitzende, lokale Wirtschaft, Genossenschaften, Vereine, Landwirtschaft, Eigenbetriebe, ...) zugehen und rentierliche Energieerzeugung-/speicher in ihrer Trägerschaft anbieten.
    \item Lokale Energieerzeugergenossenschaften sind durch Kooperation zu fördern und einzubinden.
\end{itemize}

\subsection{CO2 Neutralität bis 2035}
Die PIRATEN Dresden fordern, dass Dresden als Gesamtstadt bis 2035 klimaneutral sein muss.


\subsection{Ökofernwärme}
Wir fordern eine nachhaltige Fernwärme. Dies bedeutet, dass jegliche Wärme CO2-neutral erzeugt werden muss. Neben der kompletten Dekarbonisierung der Hauptwärmeerzeuger der SachsenEnergie ist hierzu Prozesswärme einzubinden, dezentrale Wärmespeicherung zu forcieren und die denzentrale Wärmeerzeugung (z. B. Solarthermie, Geothermie) auszubauen. Insellösungen für bislang nicht an das Fernwärmenetz angeschlossene Stadtbezirke sind ebenfalls zu fördern. Hierzu sind Partnerschaften mit privaten Immobilienbesitzenden, Genossenschaften und der lokalen Wirtschaft zu schließen.



\subsection{Energiespeicher}
Wir fordern den Ausbau von Energiespeichern in Dresden.


\subsection{Solarnutzung ausbauen}
Die PIRATEN Dresden fordern, die solare Energienutzung auf allen Dresdner Dächern massiv auszubauen (Photovoltaik und Solarthermie).


\subsection{Dach- und Fassadenbegrünung}
Die PIRATEN Dresden fordern eine verstärkte Dach- und Fassadenbegrünung sowohl bei Neu- als auch Altbauten. Um dieses Ziel gesteuert zu erreichen, kann eine Begrünungssatzung sinnvoll sein.


\subsection{Förderung von Balkonkraftwerken}
Die PIRATEN Dresden fordern, dass Balkonkraftwerke in Dresden für jeden Haushalt mit bis zu 200Euro pro erstem Balkonkraftwerk gefördert werden.

\subsection{Schwammstadt}
Die PIRATEN Dresden fordern, dass die Konzepte der 'Schwammstadt' bei allen kommunalen und privaten Bauvorhaben Anwendung findet.

\subsection{Fortführung des Müllkonzepts für den Alaunpark}
Wir fordern die konsequente Fortführung des jetzigen Müllkonzepts mit ausreichend Papierkörben, Müllcontainern und deren häufige Leerung sowie regelmäßiger Reinigung des Platzes. Hierfür sind in jedem Jahr ausreichend finanzielle Mittel (ca. 30.000 Euro) bereitzustellen. In Zukunft sollen weitere kreative Ideen (z. B. Müllsackspender, gemeinsame Sammelaktionen) die Akzeptanz für das Konzept weiter steigern. Eine Rückkehr zu Strafen und Kontrollen lehnen wir entschieden ab.


\subsection{Bepflanzung kommunaler Grünanlagen $-$ Urban Gardening}
Die PIRATEN Dresden unterstützen jegliche Form des gemeinschaftlichen Gärtnerns in der Stadt $-$ auch in kommunalen Grünanlagen. Ob Baumscheibenpatenschaften, Urban-Gardening-Anlagen, Initiativen für eine ``Essbare Stadt''. Alles, was die Stadt grüner macht und Menschen zusammenbringt, muss ermöglicht und gefördert werden.


\subsection{Keine Steingärten}
Die PIRATEN Dresden lehnen 'Steingärten', als egoistisch-kostensparende Option sich jeglicher ökologischen Verantwortung für das eigene Umfeld zu entziehen, ab.


\subsection{Stadt der erneuerbaren Energien}
Die PIRATEN Dresden fordern, dass alle kommunalen Einrichtungen (DVB, Schulen, Bäder usw.) bis 2030 ihre Wärme- und Stromversorgung aus regenerativen Energiequellen beziehen.


\subsection{Position zu Windrädern in Dresden}
Die PIRATEN Dresden fordern, das Verbot von Windkraftanlagen (Windrädern) auf dem Stadtgebiet Dresdens aufzuheben.


\subsection{Recycling-Baustoffe}
Die PIRATEN Dresden setzen sich für eine intensive Nutzung von Recycling-Baustoffen $-$ z. B.~für kommunale Gebäude $-$ ein, um Ressourcen zu schonen und Transportwege zu sparen.


\section{Jugend und Bildung}

\subsection{Hygieneartikel an Schulen und in öffentlichen Einrichtungen}
Die PIRATEN Dresden fordern, dass Hygieneartikel an Schulen und allen Öffentlichen Einrichtungen kostenlos zur Verfügung gestellt werden.


\subsection{Kita- und Hortangebote für alle Kinder}
Die Dresdner PIRATEN fordern, dass die KiTa-Platzgarantie überall wohnortnah und ausnahmslos erfüllt wird. Dazu muss die Sanierung und der Ausbau der KiTa-Plätze weiter verstärkt werden.


\subsection{Lernmittelfreiheit}
Die PIRATEN Dresden fordern eine vollständige Lernmittelfreiheit für Dresden.


\subsection{Übernachtungsmöglichkeit für schulische Exkursionen}
Die PIRATEN Dresden fordern, dass Übernachtungsangebote der Stadt bzw. deren Einrichtungen eingeführt werden, um Kinder- und Jugendgruppen auf Bildungs-Exkursionen eine (kostenfreie) Übernachtungsmöglichkeit bereitzustellen. Hierzu können z. B. Turnhallen, Klassenräume oder Versammlungsräume genutzt werden. Es soll hierbei geprüft werden, wie und in welcher Form die Gruppen sich für dieses Angebot bedanken können $-$ z. B.~ durch die Einladung einer Dresdner Gruppe in die eigene Schule bzw. dem Veröffentlichen eines Erlebnisberichts in der Schulzeitung.


\subsection{Nutzung der Flächen unter den Elbbrücken }
ie PIRATEN Dresden setzen sich dafür ein, dass die Flächen unter den Dresdner Brücken für zum Beispiel Outdoor-Sport oder für legale Graffiti-Flächen genutzt werden. Kooperationen mit Vereinen und Initiativen sind zu begrüßen.


\subsection{Ausreichende IT-Betreuung unserer Schulen}
Die PIRATEN Dresden fordern, dass die Betreuung der IT-Infrastruktur (Hard- und Software) federführend durch die Stadt koordiniert und damit auch endlich sichergestellt wird.


\subsection{Planungssicherheit für soziale Kinder- und Jugendarbeit}
Die PIRATEN Dresden setzen sich für mehrjährige Förderung von Kinder- und Jugendeirichtungen sozialer Arbeit ein, um deren Arbeit zu verstetigen und ihnen mehr Planungssicherheit zu verschaffen.


\subsection{Essen für jedes Kita- und Schulkind sichern}
Die PIRATEN Dresden setzen sich dafür ein, dass für jedes Kita- und Schulkind eine Mittagsversorgung gesichert ist. Ein veganes und vegetarisches Angebot muss dabei gegeben sein.



\subsection{Keine Werbeveranstaltungen der Polizei an Schulen}
Die PIRATEN Dresden lehnen Werbeveranstaltungen für die Polizei an Schulen ab. Oftmals vermitteln solche Veranstaltung ein verzerrtes Bild der Polizei, ignorieren strukturelle Probleme innerhalb der Polizei und lassen eine diffenzierte Diskussion nicht zu.


\subsection{Keine Bundeswehr an Schulen}
Die PIRATEN Dresden lehnen Veranstaltungen oder Werbung der Bundeswehr an Schulen generell ab.


\subsection{Schulnetzplanung}
Die PIRATEN Dresden fordern, dass bei der Fortschreibung des Schulnetzplans nicht die geringste Prognosezahl der Schüler·innen als Grundlage genutzt wird.



\subsection{Effiziente Schulnutzung}
Schulische Räume (Klassenräume, Aulas, Sporthallen, Schulhof, etc.) sollen außerschulischen Aktivitäten (Vereine, Initiativen) unbürokratisch und kostengünstig zur Verfügung gestellt werden.


\subsection{Stärkung der Volkshochschule}
Die PIRATEN Dresden sehen die Volkshochschule als wichtige und niederschwellige Bildungsmöglichkeit an. Das Angebot der Volkshochschule sollte daher stetig erweitert werden, wobei es für alle Menschen kostenlos sein muss.


\section{Soziales und Wohnen}

\subsection{Krankenhäuser in kommunaler Trägerschaft}
Die PIRATEN Dresden setzen sich für die Beibehaltung der kommunalen Trägerschaft der Dresdener Krankenhäuser ein. Doch auch in kommunalen Einrichtungen darf es keine Unterbezahlung, Überbelastung und Ausbeutung des Personals geben. Die Stadt muss finanzielle Mittel bereitstellen, um eine menschenwürdige Krankenversorgung zu ermöglichen.


\subsection{Fahrender Ritter}
Die Landeshauptstadt Dresden soll in Zusammenarbeit mit humanitären Organisationen einen Kältebus betreiben. Die PIRATEN Dresden fordern die Bereitstellung eines Kältebusses für den Zeitraum von November bis März im Stadtgebiet Dresden. Wir setzen uns dafür ein, dass diese Maßnahme nur vorübergehend notwendig ist und langfristige Lösungen gefunden werden. Niemand sollte frieren müssen!



\subsection{Pfand gehört daneben}
Die PIRATEN Dresden setzen sich für eine Umrüstung weiterer Papierkörbe zu ``pfandfreundlichen Papierkörben'' ein.


\subsection{Kein Freiheitsentzug für Fahren ohne gültigen Fahrschein}
Wer beim Fahren ohne Fahrschein erwischt wird, ist durch Freiheitsentzug bedroht. Die PIRATEN Dresden fordern daher, dass die DVB AG auf Strafanträge nach §265a StGB (``Erschleichen von Leistungen'') verzichtet.


\subsection{Inklusives Wohnen}
Die PIRATEN Dresden fordern, inklusives Wohnen (z. B. für Menschen mit Behinderungen, Senior·innen) bei allen städtischen sowie privaten Baumaßnahmen zu ermöglichen.


\subsection{``Wohnen in Dresden'' $-$ WoBa stärken}
Die PIRATEN Dresden bekennen sich zum Erhalt und Ausbau der städtischen Wohnbaugesellschaft ``Wohnen in Dresden'' (WiD). Der Wohnungsbestand ist zügig zu erweitern. Ziel sind mindestens 10.000 Wohnungen bis 2030. Hierbei sehen wir die Stadt in der Pflicht. Eine gute Durchmischung in den Stadtteilen soll dabei erhalten bleiben oder erzielt werden.


\subsection{Rückkauf von Vonoviawohnungen}
Die PIRATEN Dresden setzen sich dafür ein, Objekte und Liegenschaften zur Erweiterung des Wohnungsbestands der kommunalen Wohnungsgesellschaft (WiD) zu kaufen. Hierfür kommen insbesondere Teile des Portfolios der VONOVIA in Betracht.


\subsection{Alternative Wohnformen umsetzen}
Die PIRATEN Dresden setzen sich für die Umsetzung alternativer Wohnformen ein. Hierzu gehören z. B. Wächterhäuser, Wagenplätze, Wohninitiativen, Genossenschaften und Miethaussyndikate.


\subsection{Genossenschaftliches Wohnen fördern}
Die PIRATEN Dresden setzen sich dafür ein, dass kommunale Liegenschaften an Genossenschaften und Miethaussyndikate über Erbpacht zur Nutzung gegeben werden.


\subsection{Sozialtarif bei (Energie-)Versorgern}
Die PIRATEN Dresden fordern, dass die Stadt Dresden bei den Versorgern, an denen sie beteiligt ist, Sozialtarife für alle Energie- und Medienformen (Strom, Gas, Wasser, Fernwärme) einführt. Auf folgenschwere Strom- und Gassperren soll vollständig verzichtet werden.




\subsection{Menschlichere Bürgergeld-Verwaltung in Dresden}
Die PIRATEN Dresden streben ein Bedingungsloses Grundeinkommen als Ersatz für das ungerechte und ineffiziente Bürgergeld (ehemals Hartz IV/ALG2) an. Als einen ersten praktikablen Schritt hierzu werden wir alle kommunalen Möglichkeiten ausschöpfen, um den Bürgergeld-Bezug sanktionsfrei zu gestalten. Des Weiteren muss die Bürgergeld-Verwaltung menschlicher werden, d.h. den Fokus auf Hilfe und Unterstützung und nicht auf Kontrollen und Bestrafung legen.


\subsection{Sozialtarif bei 49-Euro-Ticket}
Die PIRATEN Dresden fordern, dass es für alle Tarife der DVB auch Sozialtarife gibt. Auch muss es eine Möglichkeit für Menschen mit negativem SCHUFA-Eintrag oder ohne Girokonto geben, Abo-Monatskarten bzw. ein 49-Euro-Ticket (Deutschlandticket) zu erwerben.


\section{Asyl und Migration}

\subsection{Herz statt Hetze}
Der gesellschaftliche Diskurs beim Thema Flucht und Asyl wird immer rauer. Vorurteile, Hass und Hetze werden geschürt und versuchen die Herzen zu vergiften. In dieser Situation ist es erst recht wichtig, es klar auszusprechen: Menschen in Not muss man helfen!\newline

Es ist daher unsere Pflicht als Stadtrat, Menschen in Not Obdach, Nahrung, Medizin und ein lebenswürdiges Auskommen zu sichern. Es ist unsere Pflicht als Stadtgesellschaft, Menschen in Not ein gutes Ankommen in Dresden zu ermöglichen. Und es ist unsere Pflicht als Menschen, unsere Herzen nicht zu verschließen und der Hetze keinen Raum zu geben.\newline

Wir PIRATEN streben eine solidarische Welt ohne jegliche Grenzen an!\newline

\#RefugeesWelcome


\subsection{Migration und Integration}
Wir begreifen Migration als Chance für den Menschen und für unsere Gesellschaft. Damit Migration gelingen kann, sind Anstrengungen aller, auch der Politik, notwendig. Hier sehen wir vor allem die Kommunen als zentrale Akteure in der Verantwortung, die Aufgaben der Integration zu übernehmen. Zu diesen Aufgaben zählen unter anderem Schul- und Kitabesuche, kulturelle Angebote und die Einbindung in den Arbeitsmarkt. Damit die Stadt Dresden sowie jede andere Kommune diese Aufgaben entsprechend erfüllen kann, ist die Unterstützung seitens Bund und Land notwendig. Wir fordern außerdem ein kommunales Integrationszentrum, in dem wichtige Schlüsselstellen (Beratung und Hilfestellungen, Sprachkurse, soziale Träger, Vereine, Anlaufpunkte für Ehrenamt) kommunal gefördert und dauerhaft finanziell abgesichert werden, damit Migrant·innen mit diesen zusammenarbeiten können.\newline

Als weiteren wichtigen Bestandteil von Integration sehen wir politische Beteiligung. Deshalb fordern wir eine Senkung der Hürden für die Teilnahme an Kommunalwahlen für EU-Bürger·innen. Des Weiteren soll der Ausländerbeirat mehr Mitbestimmungsrecht erhalten, um seine beratende Funktion auszubauen, solange es kein gleichberechtigtes Wahlrecht für alle in Dresden lebenden Menschen gibt.


\subsection{Humane Politik für Geflüchtete}
Wir PIRATEN Dresden bekennen uns ganz klar zur Initiative ``Dresden als Sicherer Hafen'' und kämpfen gegen jeden Versuch, diese Entscheidung rückgängig zu machen. Die Stadt Dresden hat mehr Platz und Kapazitäten und kann mehr Geflüchtete als rechtlich vorgeschrieben aufnehmen.\newline

Die PIRATEN Dresden fordern außerdem, Geflüchtete vor Ort durch entsprechende Maßnahmen verstärkt zu unterstützen, damit sie ihre Potentiale und Fähigkeiten entfalten können. Hierzu zählen für uns eine dezentrale Unterbringung, gesicherte Gesundheitsversorgung, existenzsichernde Mittelversorgung, Schul- und Kitabetreuung ab Ankunftstag, kulturelle Teilhabe und barrierefreie Beratung. Hierzu ist ein dauerhaft finanziertes ``Kommunales Integrationszentrum'' notwendig, das Kompetenzen und Angebote abbildet und absichert. Durch all diese Maßnahmen soll verhindert werden, dass Geflüchtete in soziale oder gesellschaftliche Schieflagen geraten. Die Würde jedes Menschen ist unantastbar, nicht nur jene von Menschen mit europäischem Pass.








\subsection{Seebrücke}
Die PIRATEN Dresden fordern die finanzielle, logistische und rechtliche Unterstützung von Vereinen und Initiativen, wie z. B. ``Mission Lifeline''. Seenotrettung darf nicht kriminalisiert werden. Wer in Not ist, muss gerettet werden. Wer nicht rettet, verabschiedet sich von der Menschlichkeit.


\subsection{Sicherer Hafen}
Die PIRATEN Dresden bekennen sich vollen Herzens dazu, Dresden als ``Sicheren Hafen'' zu deklarieren. Dieser reinen Deklaration müssen aber auch konkrete Taten folgen. Menschen in Not sind in Dresden willkommen!


\section{Kultur}

\subsection{Grundverständnis von Kulturpolitik}
Kultur ist mehr als Semperoper und Kulturpalast. Kultur ist überall dort vorhanden, wo Menschen etwas Kreatives entstehen lassen. Folgerichtig müssen in einer Kulturstadt auch Raum und Möglichkeit geboten werden, jegliche Art von Kultur auszuleben, auszuprobieren und umzusetzen. Diese flächendeckende Vielfalt und der Wunsch vieler Menschen, die Kultur in der Stadt stetig zu erweitern, machen eine Kulturstadt aus. Wir wollen jegliche Form von Kultur ermöglichen und den Zugang zu dieser erleichtern. Niemand darf von Kulturangeboten ausgeschlossen werden $-$ alle sollen sich beteiligen können.



\subsection{Kulturschutzzone}
Die PIRATEN Dresden setzen sich dafür ein, dass die Äußere Neustadt zur ``Kulturschutzzone'' erklärt wird. In dieser Kulturschutzzone ist das ausdrückliche Ziel, ein lebendiges Nachtleben zu ermöglichen.


\subsection{Mehr legale Graffitiwände}
Die PIRATEN Dresden fordern, dass zusätzliche Wände von öffentlichen Gebäuden oder eigens dafür errichtete Wände in Dresden für Graffiti freigegeben werden. Kreatives Potential ist zu binden und zu fördern. Graffitikunst ist eine anerkannte, bereichernde Kulturform.


\subsection{Kulturentwicklungsplan}
Die PIRATEN Dresden setzen sich dafür ein, den Kulturentwicklungsplan weiterhin regelmäßig und kontinuierlich fortzuschreiben. Der Plan soll ausgewogen und ganzheitlich alle Bereiche der Kunst und Kultur betrachten. Gegebenenfalls sind Ungleichgewichte zugunsten einer Sparte aufzuheben. Insbesondere bei der Freien Szene sind die Bedarfe zu erheben und zu beachten. Vorgaben von Kunstverbänden zu Gehältern und Honoraren sind zu beachten. Im Arbeitsfeld von Kunst und Kultur darf es keine prekäre Beschäftigung geben.



\subsection{Kulturförderung der sorbischen Minderheit}
Um der Situation der sorbischen Minderheit in Sachsen Rechnung zu tragen, liegt ein besonderes Augenmerk auf dem Erhalt sorbischer Kulturgüter. Dresden als Landeshauptstadt unterstützt dabei sorbische Institutionen und Vereine in ganz Sachsen bei diesem Anliegen und bietet Möglichkeiten, sorbische Kulturgüter sowohl in Dresden als auch über die Lausitz hinaus bekannt zu machen.


\subsection{Straßenkunst}
Straßenkunst und Straßenmusik sind eine wundervolle Art, eine lebendige und lebensfrohe Stadt zu erhalten. Daher fordern die PIRATEN Dresden, Straßenkunst nach einfachen und transparenten Regeln im gesamten Stadtgebiet zu ermöglichen.



\subsection{Kulturbildung barriere- und kostenfrei gestalten}
Die PIRATEN Dresden wollen $-$ parallel zur Schulbildung $-$ die außerschulische kulturelle Bildung insbesondere für Kinder und Jugendliche sichern. Das umfasst alle staatlichen Bildungsmöglichkeiten in Tanz, Theater, Musik, Sport und allen anderen Formen bildender und darstellender Künste. Die individuelle Förderung für die Bildung an privaten Einrichtungen soll im Bedarfsfall ebenfalls möglich sein. Ein erster Schritt in diese Richtung ist der für Kinder und Jugendliche kostenlose Eintritt in alle Museen der Stadt, zumindest an einem Tag in der Woche.


\section{Sport}

\subsection{Zweck und gesellschaftlicher Mehrwert des Sports}
Ziel der PIRATEN Dresden ist es, die Sportinfrastruktur in gutem Zustand zu erhalten und zu ergänzen. Wir verstehen aber unter Sportinfrastruktur nicht nur die auf den Vereinssport abzielenden Stadien und Hallen, sondern auch Halfpipes, Skatebahnen, Joggingpfade, Bike-Trails, öffentliche Schachbretter, Basketballkörbe und Bolzplätze. Gerade diese niederschwelligen, öffentlichen und auf private Eigenvernetzung abzielenden Sportarten sind zu fördern.



\subsection{Position zum E-Sport}
Die PIRATEN Dresden fordern, dass E-Sport als offizielle Sportart anerkannt und in die Sportförderrichtlinie aufgenommen wird. Aus wissenschaftlicher Sicht befindet sich der E-Sport bereits heute auf dem Niveau einer traditionellen Sportart. Die unterschiedlichen Spiele weisen eine hohe Komplexität auf und schulen kognitive Fähigkeiten; auch der Aspekt der Fitness gewinnt immer mehr an Bedeutung.\newline

Die PIRATEN Dresden fordern, dass die Stadt eine Vorreiterrolle bei der Einführung eigener E-Sport Stadtligen sowie bei der Veranstaltung von (internationalen) E-Sport-Events einnimmt.


\section{Verkehr}

\subsection{DVB als zentrale Mobilitätsdienstleisterin}
Die PIRATEN Dresden werden die DVB AG zur zentralen Mobilitätsdienstleisterin in Dresden ausbauen. Die Taktung der Busse muss erhöht, das Liniennetz verstärkt, die Qualität weiter verbessert und die Mobi-Welt um weitere Sharing-Angebote ausgebaut werden. Hierzu muss die Finanzierung der DVB AG dauerhaft und nachhaltig gesichert sein, damit Investitions- und Betriebsmittel planungsicher zur Verfügung stehen. Hierfür muss neben den abnehmenden Fahrscheineinnahmen (s.``0-Euro-Ticket''), den Zuschüssen aus Land und Bund und den Zuwendungen aus den Gewinnen der SachsenEnergie eine weitere Finanzierungssäule für die DVB dauerhaft installiert werden.\newline

Diese weitere Finanzierungsssäule sollte nicht dauerhaft ein Zuschuss aus dem allgemeinen Haushalt sein, da Haushaltsmittel immer alle zwei Jahre wieder neu verhandelt werden und somit der DVB keine echte Planungssicherheit geben können. Eine verlässliche und langfristige Finanzierungssäule wird die sogenannte Nutznießerfinanzierung [1] darstellen, bei der von der DVB (mittelbar) Profitierende über Abgaben ihren Beitrag zur ÖPNV-Finanzierung leisten werden (z. B. Arbeitgeber·innen, Parkgebühren).



\subsection{Förderung des Fahrradverkehrs}
Der Fahrradverkehr ist eine umweltfreundliche und ressourcenschonende Verkehrsart und muss deshalb gefördert werden. Zusätzlich ist in einer sich verdichtenden Stadt der Radverkehr ideal geeignet, den knapper werdenden Raum effizient zu nutzen $-$ dort wo heute noch ein Auto parkt, können morgen 20 Fahrräder abgestellt werden. Um den Radverkehr substanziell fördern zu können, sind die Ausgaben für den Radverkehr im Haushalt schnellstmöglich auf mindestens 20,00 Euro pro Einwohnenden und Jahr zu erhöhen. Damit kann das Radverkehrskonzept der Stadt zügig und ohne Abstriche umgesetzt werden.\newline

Über das Radverkehrskonzept hinaus sehen wir deutlichen Handlungsbedarf, den Radverkehr schneller, sicherer und komfortabler zu gestalten.

Die PIRATEN Dresden fordern:
\begin{itemize}
    \item In Dresden ist ein zusammenhängendes und die Stadtteile verbindendes Netz von Fahrradstraßen einzuführen. Auf Fahrradstraßen haben Radfahrende Vorfahrt und besonderen Schutz. Auch können Radfahrende nebeneinander fahren. Autos können auf Fahrradstraßen zugelassen werden, sie müssen dann allerdings besonders Rücksicht nehmen.

    \item  Neben den beiden bereits geplanten Fahrradparkhäusern an den Bahnhöfen ist sicheres und komfortables Fahrradparken flächendeckend einzuführen. Dies bedeutet neben des klassischen Fahrradbügels auch überdachte Abstellmöglichkeiten sowie abschließbare Fahrradboxen. Überall, wo Quellen und Ziele des Radverkehrs sind, muss es Fahrradabstellmöglichkeiten in ausreichender Qualität und Quantität geben.

    \item  Wir fordern Radschnellwege vom Stadtrand in die Innenstadt und zwischen den Stadtteilen. Diese Radschnellwege können baulich abgetrennt sein oder z. B. durch eine ``Grüne Welle für Radfahrer'' realisiert werden.

    \item  Wir fordern Investitionen in neue Bauprojekte, die hohe Barrieren für den Radverkehr $-$ und damit die Reisezeit $-$ stark reduzieren. Beispiele hierfür sind eine kreuzungsfreie Überquerung der Stauffenbergallee für einen Radschnellweg zwischen Klotzsche und der Neustadt sowie eine Fahrradbrücke über die Elbe, welche Pieschen/Trachenberge mit der Messe, dem Ostragehege und der Innenstadt/Friedrichstadt verbindet.

    \item  Der Winterdienst für Radfahrende muss erweitert werden.

    \item Die über 200 Unfallschwerpunkte für Radfahrende sind schnellstmöglich zu beseitigen.
\end{itemize}


\subsection{Stadt der kurzen Wege $-$ Walkable City}
Die städtebauliche Entwicklung hat sich daran auszurichten, dass eine gute Durchmischung der Stadtbezirke erzielt wird. So vielfältig wie die Menschen sollen auch die Stadtbezirke sein. Die Versorgung des täglichen Bedarfs und Freizeitmöglichkeiten sollen überall in fußläufiger Entfernung angeboten werden. Ebenso sollen öffentliche Einrichtungen wie Schulen, Kitas, Gemeindeämter und Kultureinrichtungen vorzufinden sein. Ziel ist eine kompakte Stadt, in der die meisten Wege zu Fuß oder mit dem Fahrrad erledigt werden können.\newline

Auf dieses Ziel arbeiten die PIRATEN Dresden mit folgenden Mitteln hin:

\begin{itemize}
    \item Kitas, Horte und Grundschulen müssen flächendeckend eine dezentrale Versorgung sicherstellen.
    \item Die Stadtteilzentren sind baulich und funktional zu stärken $-$ Aufenthalt, Barrierearmut, Einkaufen, Begegnung und verkehrliche Entschleunigung sind die Hauptfunktionen der Stadtteilzentren.
    \item Kleine Parks und Grünzonen sind neu einzuplanen und bei Möglichkeit zu vernetzen.
    \item Bei (privaten) Bebauungsplänen fordern wir die Verantwortung der Investoren für das Allgemeinwesen ein. Wer baut, muss sich auch an der allgemeinen Entwicklung der Stadt (baulich, sozial, finanziell) beteiligen.
    \item Eine kompakte, mehrgeschossige Bauweise in Kombination mit Grünflächen und angenehmen Innenhöfen ist in der Stadt wünschenswert. Neubauten von kleinen Einzelhäusern sind hingegen kritisch zu sehen.
    \item Neubauflächen am Stadtrand für Einfamilienhäuser sollen zwingend an den ÖPNV angeschlossen sein.
    \item Stadtbäume sollten groß werden können. Dies ist insbesondere bei der Planung von unterirdischen Leitungen und Tiefgaragen zwingend zu beachten.
    \item Das Konzept des ``Shared Space'' ist bei Neubauten und Umgestaltung von Straßenräumen generell zu prüfen. Bis 2025 soll mindestens ein weiterer Straßenzug im Sinne des ``Shared Space'' umgestaltet werden.
    \item Wir begrüßen Fassaden- und Dachbegrünung.
\end{itemize}

\subsection{Förderung des Fußverkehrs}
Die PIRATEN Dresden fordern, dass verstärkte Aufmerksamkeit der Schaffung und Erhaltung von sicheren und attraktiven Fußwege-Beziehungen gewidmet wird. Bei Neubauten von Gebäuden, Straßen und Plätzen müssen die Fußwege und die entstehenden Wegebeziehungen vorrangig betrachtet werden. Fußläufige Wege sollten sich auf natürliche Art und Weise ergeben und eine möglichst direkte, sichere und komfortable Verbindung ermöglichen. Das städtebauliche Ziel der ``kompakten Stadt'' ist nicht im Sinne von ``wir wollen maximal dicht bauen'' zu interpretieren, sondern vielmehr als ``wir wollen eine möglichst hohe Dichte an fußläufigen Quellen und Zielen''. Dies bedeutet, dass in jedem Stadtteil Wohnen, Arbeiten, Einkaufen und Freizeit möglichst kleinteilig und fußläufig angeboten werden muss.



\subsection{Autofreie Äußere Neustadt}
Die PIRATEN Dresden streben das Ziel der ``Autofreien Äußeren Neustadt'' an. Ähnlich der Superblocks in Barcelona wird der heute durch (parkende) Autos ineffizient genutzte Stadtraum dadurch öffentlich nutzbar $-$ ``Reclaim the Street'' wird Wirklichkeit.\newline

Als erste Schritte in diese Richtung fordern wir:

\begin{itemize}
    \item Parkverbot in Louisen-, Alaun-, Kamenzer, Rothenburger und Görlitzer Straße sowie am Martin-Luther-Platz.
    \item  Einführung von Stellplätzen für Lastenräder/E-Bikes (inkl. Ladesäulen).
    \item  Nutzung des freigewordenen Raums zum Flanieren, für Radbügel, Stadtgrün und für den Einzelhandel sowie die Gastronomie.
    \item  Freigabe des Straßenraums für den Fußverkehr.
    \item  Verhinderung der Durchquerbarkeit der Äußeren Neustadt für den KFZ-Verkehr.
    \item  Massive Erhöhung der Parkgebühren für die noch verbleibenden Parkplätze.
    \item  Einführung eines elektrischen Quartierbusses zur Anbindung der Parkhäuser in der Peripherie.
    \item  Ausweisung der Parkplätze an den Außengrenzen (z. B. Königsbrücker, Bischofsweg, Bautzner) für Car-Sharing Angebote.
    \item Bike-Sharing-Angebot im gesamten Viertel.
    \item Parkplätze für Schwerbehinderte müssen innerhalb der Neustadt beibehalten und ausgeweitet werden, bis das Gesamtkonzept ``Autofreie Neustadt'' komplett barrierefrei umgesetzt wurde.
\end{itemize}


\subsection{Park $\&$ Ride ausbauen}
Die PIRATEN Dresden setzen sich für einen Ausbau von Park $\&$ Ride-Angeboten in Dresden ein. An allen Pendlerstrecken sind benutzungsfreundliche Parkplätze zu schaffen, die den Umstieg in leistungsfähige öffentliche Nahverkehrsträger sowie auf (Leih-)Fahrräder ermöglichen.


\subsection{Bessere Radumleitungen}
Die PIRATEN Dresden setzen sich für eine alltags- und familientaugliche Radwegeführung bei Baustellen und Veranstaltungen ein. Dafür müssen kommerzielle Veranstalter und Baulastträger ggf. einen finanziellen Beitrag leisten. Um sichere und komfortable Umleitungen erhalten zu können, müssen bei Bedarf auch Kfz-Fahrspuren reduziert werden.


\subsection{``Kostenloses'' Parken ist kein Grundrecht}
``Kostenloses'' Parken von Autos im öffentlichen Raum ist in den letzten Jahrzehnten derart weitverbreitet gewesen, dass viele Autobesitzende glauben, ein Auto kostenlos abzustellen wäre ein Grundrecht. Dies ist aber nicht so, denn es gibt kein Grundrecht auf kostenloses Parken! Der öffentliche Raum gehört uns allen $-$ und alle müssen für diesen Raum bezahlen.

Die PIRATEN Dresden fordern, dass Parken im öffentlichen Raum grundsätzlich kostenpflichtig sein muss. Hierfür soll ein Mix aus Anwohnerparkzonen und Parkraumbewirtschaftung sorgen, welcher natürlich Elemente der sozialen Abfederung sowie Härtefallregelungen enthalten muss.


\subsection{Lastenrad Lieferdienste}
Die PIRATEN Dresden fordern, Lieferdienste, die Waren mit Lastenrädern transportieren, zu fördern.


\subsection{Städtischer Zuschuss zu Lastenrädern}
Die PIRATEN Dresden setzen sich dafür ein, dass juristischen und natürlichen Personen Anreize für den Kauf von Lastenrädern gewährt werden. Haushaltsmittel der Stadt, Sponsorenmodelle und Förderprogramme sind hierbei zu prüfen.


\subsection{Lastenräder in die Mobi-Welt integrieren}
Die PIRATEN Dresden fordern, das Angebot an Leih-Lastenrädern im Rahmen der Mobi-Welt weiter auszubauen.


\subsection{Bike-Sharing-Konzepte}
Die PIRATEN Dresden unterstützen Bike-Sharing-Konzepte, die datensparsam und mit möglichst geringem technischem Aufwand nutzbar sind. Anbieter, deren Geschäftsmodell neben dem eigentlichen Fahrradverleih auf dem Handel mit den Nutzungsdaten basiert, lehnen wir ab. Wünschenswert ist der Ausbau von Lastenräder-Leih-Möglichkeiten.


\subsection{Radbügel jetzt}
Die PIRATEN Dresden setzen sich dafür ein, dass in dicht besiedelten Gebieten (z. B. Äußeren und Inneren Neustadt, Hechtviertel, Löbtau, Pieschen, Striesen, Blasewitz) alle Wohn- und Geschäftsadressen in einem Abstand von höchstens 50 m Fahrradabstellanlagen (z. B. Radbügel) in ausreichender Anzahl vorhanden sein müssen.


\subsection{Interkommunale Zusammenarbeit im Radverkehr}
Die PIRATEN Dresden fordern, den interkommunalen Radverkehr zwischen Dresden und den Gemeinden im Umland zu fördern. Ziel ist, Dresden in ein Netz von Fahrradfernwegen zu integrieren.


\subsection{Zebrastreifen}
Die PIRATEN Dresden fordern eine Zebrastreifen-Offensive: An möglichst vielen Stellen sollen Zebrastreifen angebracht werden.


\subsection{Carsharing}
Die PIRATEN Dresden fordern eine konsequente, aber durchdachte Förderung von Carsharing- Angeboten. Das primäre Ziel von Carsharing ist eine Reduktion der Gesamtanzahl an Pkw und gleichzeitig eine Verlagerung des heutigen Autoverkehrs auf umweltfreundlichere Verkehrsarten. Das geliehene Auto ist dann die ideale Ergänzung, um gelegentliche Autofahrten zu ermöglichen. Daher sollten neue Carsharing-Stationen in der Nähe von Haltestellen (Mobilitätsstation) bzw. in der Nähe der Wohnorte entstehen.\newline

Bei Free-Floating Carsharing-Modellen muss deren Auswirkung auf den Gesamt-Autoverkehr kritisch hinterfragt werden. Es ist nicht Ziel des Carsharing, in Konkurrenz zu Fuß-, Rad- und ÖPN- Verkehr zu treten. Für Free-Floating Modelle sollten keine Bonifikationen wie Parkgebührreduktion oder Parkplatzsicherheit gewährt werden.



\subsection{Stadtbahn}
Die PIRATEN Dresden begrüßen den Bau einer Straßenbahnlinie entlang der völlig überlasteten Buslinie 61 von Löbtau bis Strehlen. Außerdem befürworten wir die geplante Linie von Johannstadt bis Plauen sowie eine mögliche Linienverlängerung der Linie 11 bis Weissig. Bei allen Vorhaben ist aber darauf zu achten, dass durch den Gleisbau das Stadtbild und die Funktionalität des Stadtraums nicht eingeschränkt werden. Ziel ist nicht, 20 Meter breite Schneisen durch Gründerzeitviertel zu schlagen, sondern diese Viertel mit einer Straßenbahn in das ÖPNV- Netz zu integrieren.\newline

Beim weiteren Ausbau des Stadtbahnnetzes sind auch neue Ideen ergebnisoffen zu prüfen (Oberleitungsbusse, Monorail, Seilbahn, autonom fahrende Busse, Wassertaxis, Amphibienbusse, ...).\newline

Bei allen Bauvorhaben ist eine frühzeitige und fundierte Bürger·innenbeteiligung sicherzustellen.


\subsection{Innovativer DVB-Fuhrpark}
Die PIRATEN Dresden befürworten, dass der DVB-Fuhrpark für ökologische und innovative Pilotprojekte genutzt wird. Hierzu sind aktiv Fördergelder zu akquirieren sowie auf eine gute Vernetzung zu Forschungseinrichtungen und Anbietern hinzuarbeiten.


\subsection{S-Bahn}
Die PIRATEN Dresden setzen sich für eine Stärkung des bestehenden S-Bahn-Netzes ein. Hierbei sollen Takte verdichtet und neue Verbindungen mit dem Umland geprüft werden. Zusätzlich sollen die seit Ewigkeiten geplanten neuen Haltestellen am Olbrichtplatz und an der Nossener Brücke endlich entstehen. Hier werden wir über den VVO Druck machen. Es ist zu prüfen, ob auf den zukünftig mit ETCS ausgebauten Korridoren automatisierte Verstärkerfahrten bzw. Nachtverkehr eingerichtet werden können (Pilotprojekt).


\subsection{Quartierbusse}
Die PIRATEN Dresden setzen sich für die Einrichtung von weiteren Quartierbuslinien ein. Die bislang bestehenden Quartierbuslinien sind zu erhalten bzw. zu verbessern.



\subsection{0-Euro-Ticket}
Die PIRATEN Dresden setzen sich für die Einführung eines für die Nutzenden kostenlosen, d.h. umlagefinanzierten öffentlichen Personen-Nahverkehrs (ÖPNV) ein. Ein solches ``0-Euro-Ticket'' sehen wir zusammen mit einem gut ausgebauten ÖPNV als festen Bestandteil der Daseinsvorsorge an. Solch ein ÖPNV garantiert, dass auch einkommensschwache Menschen die Möglichkeit zu mehr gesellschaftlicher Teilhabe erhalten. Außerdem verlagert ein solch attraktiver ÖPNV viel Kfz-Verkehr in Bus und Bahn.\newline

Die PIRATEN Dresden werden all ihren Einfluss über die DVB und den VVO einsetzen, dass das heutige 49-Euro-Ticket schrittweise verbilligt und schlussendlich zu einem 0-Euro-Ticket wird.


\section{Stadtentwicklung}

\subsection{Keine weiteren Hypermärkte in Dresden (Globus)}

Die Piratenpartei Dresden setzt sich für eine nachhaltige und lebenswerte Stadt der kurzen Wege ein. Wohnen, Arbeiten, Lernen, Einkaufen und Freizeit sollten in allen Stadtteilen komfortabel, ökologisch, preiswert, sicher und schnell miteinander kombinierbar sein, vorzugsmäßig zu Fuß, mit dem Rad oder mit dem ÖPNV. Hypermärkte für den Einzelhandel hingegen zementieren die Abhängigkeit vom Auto. Wertvolle Kaufkraft wird dadurch aus den Stadtteilen gezogen und dadurch eine Zentrierung des Handels und einen Autoverkehrsanstieg weiter beschleunigt. Durch die Abnahme des dezentralen Einzelhandels wird der Effekt verstärkt und immer mehr Menschen würden ihre Einkäufe zentral erledigen. Das führt letzten Endes zu einer deutlich reduzierten Lebensqualität in den Stadtteilen und insbesondere zu einer Verschlechterung der Situation für kleine inhaber·innengeführte Läden. Deshalb sind wir dagegen, dass weitere Hypermärkte in unserer Stadt eröffnet werden. Wir setzen uns stattdessen für eine Stadtplanung ein, welche kurze Wege fördert und die Lebensqualität der Bewohner·innen verbessert. \newline

Spezifisch für den geplanten Hypermarkt zwischen Bremer- und Hamburger Straße und der weiterhin bestehenden Pläne eines (bzw. mehrerer) Hypermärkte im Umfeld des Alten Leipziger Bahnhofs bedeutet das:

\begin{enumerate}

\item Es ist kein Baurecht für einen Hypermarkt in der Friedrichstadt zu schaffen. Der vorhabenbezogene Bebauungsplan ist aufzuheben (V1532/22, Vorhabenbezogener Bebauungsplan Nr. 6044, Dresden-Friedrichstadt Nr. 4, Hamburger Straße/Bremer Straße, Globus SB-Markt).
\item Ein neuer Bebauungsplan, der zwingend mit dem Zentrenkonzept in Einklang sein muss, ist für diese Fläche aufzustellen.
\item Aufhebung des vorhabenbezogenen Bebauungsplans für einen Hypermarkt auf der Fläche des Alten Leipziger Bahnhofs (V1234/11, Vorhabenbezogener Bebauungsplan Nr. 6007, Dresden-Neustadt, Globus SB-Markt am Alten Leipziger Bahnhof).
\item Aufstellung eines neuen Bebauungsplans für das gesamte Gebiet Leipziger Vorstadt (Erfurter Straße, Bahndamm, Eisenbahnstraße, Leipziger Straße) mit dem Ziel, gemeinwohlorientierter Mischnutzung [1].
\item Es sind Verhandlungen mit allen Eigentümern (Friedrichstadt, Leipziger Vorstadt) zu führen, um Flächen in kommunalen Besitz zu bekommen.
\end{enumerate}


\subsection{Keine weiteren Hypermärkte}
Die PIRATEN Dresden fordern, dass es keine Neubauten von weiteren großen Einkaufsmärkten (großflächiger Einzelhandel) in Dresden geben darf.

\subsection{Novelle der Stellplatzsatzung}
Die PIRATEN Dresden fordern, die Stellplatzsatzung dahingehend zu ändern, dass sie keine Vorgaben mehr zu einer Mindestanzahl an verbindlich zu schaffenden Pkw-Stellplätzen bei Neubauprojekten enthält.

\subsection{Elbquerung bei Pieschen}
Die PIRATEN Dresden setzen sich dafür ein, dass eine neue Elbquerung zwischen Ostragehege (Messe) und Pieschen/Mickten für Fuß- und Radverkehr errichtet wird. Bis zu deren Realisierung soll eine Fährverbindung eingerichtet werden.


\subsection{Kleingartenanlagen integrieren statt verlagern}
Die PIRATEN Dresden verstehen die Klein- und Schrebergartenanlagen als integralen Bestandteil der Stadt, die das Stadtklima verbessern und den Erholungswert steigern. Eine zwangsweise Verlagerung/Schließung der Anlagen ist abzulehnen. Um die Teilhabe aller zu ermöglichen, setzen wir uns für die Förderung offener bzw. durchlässiger Kleingartenanlagen ein.


\subsection{``Öffentlicher Service'' $-$ Trinkbrunnen, WLAN-Hotspots und Toiletten}
Die PIRATEN Dresden fordern, kostenfreie Serviceleistungen wie Trinkbrunnen, WLAN-Hotspots und Toiletten vermehrt im öffentlichen Raum zu platzieren. Darüber hinaus möchten die Dresdner PIRATEN die Idee des ``Öffentlichen Service'' im Sinne einer benutzbaren Stadt ausbauen. So können z. B. Freistrom-Litfaßsäulen, öffentliche Fahrradpumpen oder öffentliche Pizzabacköfen das Leben bereichern. Die Idee des ``Öffentlichen Service'' ist auch bei allen Bebauungsplänen mitzudenken und Elemente daraus umzusetzen.


\subsection{Königsbrücker Straße}
Ziel der PIRATEN Dresden ist eine städtebauliche Aufwertung der gesamten Königsbrücker Straße als einen funktionierenden und mit Leben gefüllten Stadtraum. Vom Albertplatz bis zur Stauffenbergallee soll die Königsbrücker Straße als Boulevard eine hohe Aufenthaltsqualität aufweisen und damit ihre Funktion als belebtes Zentrum zwischen Hechtviertel und der Neustadt zurückgewinnen. Der Fokus der Sanierung muss $-$ wie bei jedem Stadtzentrum $-$ auf den Bedürfnissen des Fuß-, Rad- und öffentlichen Verkehrs liegen. Dafür muss der Anteil des Kfz (Durchgangs-)Verkehrs deutlich reduziert werden. Mit der Waldschlößchenbrücke sowie der Hansastraße stehen dieser Verkehrsart ausreichend gute Verbindungen in den Dresdner Norden zur Verfügung. Die PIRATEN Dresden setzen sich daher weiterhin für eine Sanierung im Bestand ein! Die momentan bestehende Planung, welche auf 2/3 der Königsbrücker Straße einen vierspurigen Ausbau (zwei Kfz- und zwei Bahnspuren) vorsieht, lehnen wir ab.


\subsection{Boulevard Kesselsdorfer Straße}
Mit der Vollendung der Zentralhaltestelle Kesselsdorfer Straße (Haltestelle Tharandter Str./Löbtau Center) und der Sanierung des Abschnitts bis zur Wernerstraße ist es gelungen, einen Großteil des Durchgangsverkehrs aus dem Ortsteilzentrum herauszuhalten und über die Coventry-Straße umzuleiten. Dies ermöglicht nun eine städtebaulich hochwertige Sanierung des weiteren Verlaufs der Kesselsdorfer Straße in Richtung Westen: Der einstige Boulevardcharakter der Kesselsdorfer Straße kann wiederhergestellt werden! Breite Gehwege mit großen Bäumen bieten Aufenthaltsqualität und steigern die Attraktivität des lokalen Einzelhandels und der Gastronomie. Der Radverkehr kann sicher über eine fast autofreie und geschwindigkeitsreduzierte Straße geführt werden. Die Straßenbahnen können verzögerungsfrei über die Kesselsdorfer Straße fahren und an barrierefreien Haltestellen halten. Die Grünfläche vor dem Neuen Annenfriedhof kann erhalten und aufgewertet werden. Die PIRATEN Dresden fordern, die Kesselsdorfer auf ihrer gesamten Länge bald wieder als Boulevard erstrahlen zu lassen.


\subsection{Elektronische Werbetafeln}
Wir fordern die Abschaffung elektronischer Werbetafeln.

\subsection{Autofreier Campus}
Die PIRATEN Dresden fordern, den Campus der TU-Dresden autofrei zu gestalten.


\subsection{Autofreier Waldpark}
Die PIRATEN Dresden fordern, den Waldpark autofrei zu gestalten. Innerhalb des Waldparks soll es keinen Kfz-Verkehr auf dem Lothringer Weg und Vogesenweg mehr geben.


\subsection{Parkplätze zu Gastrobereichen}
Die PIRATEN Dresden fordern, dass Restaurants und Kneipen Parkplätze im Umfeld des Unternehmens für die Bewirtung von Gästen nutzen können.


\subsection{Spielplätze}
Spielplätze sind eine wichtige Investition in die Zukunft und müssen in gepflegtem Zustand gehalten werden. Dabei setzen wir auch auf Anwohnerinitiative und Alternativangebote für alle Interessengruppen, zum Beispiel für Jugendliche und Hundehalter, statt auf Verbote und Öffnungszeiten.


\subsection{Barrierefreie Stadt}
Die PIRATEN Dresden wollen eine Gesellschaft, an der alle teilhaben können. Deshalb sind Sprach-, Seh-, Hör- und materielle Barrieren abzubauen. Ob abgesenkte Bordsteine oder Untertitel bei Ratssitzungen, das Streben nach Barrierefreiheit ist bei jedem unserer Programmpunkte zu beachten.


\subsection{Studentisches Wohnen ermöglichen}
Die PIRATEN Dresden fordern, dass studentisches Wohnen bezahlbar zur Verfügung steht. Hierzu ist die Wohngelderteilung zu entbürokratisieren und neuer Wohnraum über die WiD bzw. im Zuge von Bebauungsplänen verbindlich zu schaffen.


\subsection{Kooperatives Baulandmodell wiederherstellen}
Die PIRATEN Dresden fordern, im Kooperativen Baulandmodell wieder den ursprünglichen Anteil an zwingend zu realisierenden Sozialwohnungen von 30\% festzuschreiben. Zusätzlich soll die Bindungsfrist nicht 15, sondern mindestens 20 Jahre betragen.


\subsection{Gentrifizierung entgegenwirken}
Die PIRATEN Dresden fordern, Gentrifizierungseffekte in Wohnquartieren frühzeitig zu ermitteln und wirksam z. B. durch Einführung von Millieuschutzsatzungen zu verhindern. Darüber hinaus ist eine gezielte Degentrifizierung anzustreben.


\subsection{Partizipativer Ansatz bei der Stadtraumentwicklung}
Bei großflächigen Stadt-Neu-Entwicklungen $-$ d. h. dem Entstehen ganzer Stadtquartiere $-$ sollte nicht ein Investor oder allein die Stadt die Entwicklungsziele und deren planerische Umsetzung ``Top-Down'' bestimmen. Vielmehr muss dieser Stadtraum, der später von Menschen akzeptiert und genutzt werden soll, auch von diesen Menschen ``Bottom-Up'' mit geplant werden. Daher sind in einem partizipativen Prozess zunächst die Notwendigkeiten und Wünsche aller möglichen Akteure (Kommune, Investoren, Genossenschaften, Vereine, Baugemeinschaften, Handwerk, Gewerbe, Einzelpersonen etc.) zu erfassen und in die Planungen zu integrieren. Diese Präferenzen sind in einem nächsten Schritt planerisch und rechtlich zu regeln. Dann erst erfolgt die Umsetzung.\newline

Ein solches Vorgehen wird stark vereinfacht, wenn zumindest Teile der Grundstücksfläche in kommunaler Hand sind. Hierzu soll die Kommune vorausschauend Grunderwerb tätigen bzw. von ihrem Vorkaufsrecht Gebrauch machen.


\subsection{Progressive Liegenschaftspolitik}
Die PIRATEN Dresden fordern, dass die Kommune ihre Verantwortung für alle Menschen auch in der Liegenschaftspolitik wahrnimmt und mittels des kommunalen Planungsrechts durchsetzt. Hierzu dienen z. B.~die gemeinwohlorientierte Vergabe von Grundstücken der öffentlichen Hand (auch Erbpacht), die gezielte Bodenbevorratung mit Zwischennutzung sowie ein verstärktes Nutzen des kommunalen Vorkaufsrechts.


\subsection{Zielvorgaben für Bebauungsplan ``Leipziger Vorstadt''}
Die PIRATEN Dresden setzen sich dafür ein, dass das Gebiet der ``Leipziger Vorstadt'' (zwischen Leipziger-/Erfurter Straße und Bahndamm) zu einem Wohn-, Kultur- und Lebensquartier entwickelt wird. Die bestehenden Nutzungen wie z. B. der ``Alte Schlachthof'', ``Blaue Fabrik'' und der ``Wagenplatz'' müssen integriert werden. Der ``Alte Leipziger Bahnhof'' ist in seiner Substanz zu erhalten und einer neuen Nutzung, z. B. als Stadtteilzentrum, zuzuführen.\newline

Beim Wohnungsbau ist auf eine soziale Durchmischung zu achten. Hierbei ist ein Anteil von mindestens 30\% Prozent des Wohnraums mit Sozialbindung umzusetzen.\newline

Besondere Bedeutung kommt der angenehmen Durchquerbarkeit des Gebiets für den Rad- und Fußverkehr zu. Hier soll eine neue Wegebeziehung zwischen Pieschen (Gehestraße) und der Neustadt sowie zwischen dem Neustädter Bahnhof und der Elbe entstehen. Entlang dieser Verbindungen ist auf ausreichend Freiraum (kleine Parks, Grüninseln) zu achten. Diese Wege sollen auch die Haupterschließungsfunktion für das Gebiet erfüllen, welches im Wesentlichen autofrei werden muss.\newline

Jeglichen großflächigen Einzelhandel $-$ egal von wem $-$ lehnen wir auf dieser Fläche ab!


\subsection{Südpark}

Die PIRATEN Dresden setzen sich für eine konsequente Umsetzung des ``Südparks'' als neuer Campuspark ein. Das gesamte Areal zwischen Passauer Str., Nöthnitzer Str., Bergstr. und Kohlenstraße muss als Erholungs- und Grünanlage erschlossen werden. Größere Bauwerke, Autoparkplätze oder gar Parkhäuser innerhalb des neuen Parks lehnen wir ab.


\subsection{Reform der Sperrgebietsverordnung}
Die PIRATEN Dresden setzen sich für eine Reform der Sperrgebietsverordnung ein. Sexarbeit ist kein Verbrechen, sondern ein legaler Beruf. Repression und Schikane führen nur zu Illegalität, geringerer Sicherheit und Ausnutzung von Menschen.


\subsection{Wir geben nicht auf: Wasser in das Sachsenbad}
Das denkmalgeschützte Sachsenbad ist sowohl aus historischer Sicht als auch gesellschaftlich von großer Bedeutung für die Menschen in Dresden. Schon mehrere Generationen haben hier das Schwimmen gelernt und dabei das Gebäude und dessen Umfeld als belebten gesellschaftlichen Treffpunkt genutzt. Der Verlust des Sachsenbades schmerzt $-$ das Sachsenbad fehlt!\newline

Die PIRATEN Dresden bedauern zutiefst, dass der Stadtrat den eindeutigen Willen des ``Bürgerforums Sachsenbad'' (über 90\% Zustimmung zu einer Nutzung als öffentliches Gesundheitsbad) missachtet und das Gebäude einem privaten Investor überlassen hat.\newline

Wir PIRATEN Dresden geben aber nicht auf! Für den Fall, dass der Investor sein Bauvorhaben nicht durchführen kann oder will, muss das Sachsenbad wieder in kommunale Hand zurückfallen. In diesem Fall werden die PIRATEN Dresden wieder ``Wasser in das Sachsenbad'' lassen. Das Sachsenbad ist zwingend in kommunaler Trägerschaft als Schwimm- und Gesundheitsbad zu sanieren und zu betreiben. Die dafür notwendigen Mittel sind im Haushalt mit Priorität zu sichern.


\subsection{Übigauer Ufer aufwerten}
Die PIRATEN Dresden fordern, den Uferbereich Übigaus (von Alt-Mickten bis zum alten Werftgelände) inklusive der historischen Zuwegungen (Treppen) und Lein- und Treidelpfade aufzuwerten, als bauhistorisches Gesamtensemble wieder auferstehen zu lassen und durch moderne Nutzungen (Freizeit, Gastronomie, Kultur) einem echten Mehrwert für die Dresdner·innen zuzuführen.


\section{Wirtschaft/Finanzen}

\subsection{Erhalt der Spätshops}
Die PIRATEN Dresden sprechen sich gegen jegliche Einschränkung des Straßenverkaufs der Spätshops aus. Insbesondere darf kein erneutes Verkaufsverbot in der Polizeiverordnung verankert werden.


\subsection{Sachsenenergie als Glasfaseranbieterin}
Die PIRATEN Dresden unterstützen die Bestrebungen der SachsenEnergie, flächendeckend Internetanbieterin zu werden. Als Beitrag zur Kundenbindung und der Daseinsvorsorge soll die SachsenEnergie zu jedem (Strom, Gas, Wasser, ...) Anschluss auch eine kostenlose Mindest- Internetverbindung anbieten.


\subsection{Position zur ``Bettensteuer''}
Die Piratenpartei Dresden befürwortet grundsätzlich die ``Bettensteuer'' (Beherbergungssteuer) aus folgenden Gründen:

\begin{itemize}

    \item Dresden gibt viel Geld für Kultur, Infrastruktur und sonstige touristische Angebote aus $-$ hierfür sind die Einnahmen aus der Bettensteuer eine akzeptable Kompensation.

    \item Die jetzigen Einnahmen in Höhe von über 10.000.000.- Euro jährlich sind fest im Haushalt eingeplant. Wer die Bettensteuer abschaffen möchte, muss daher diesen Betrag an anderer Stelle einsparen bzw.~ andere Steuern (Grundsteuer, Gewerbesteuer) oder Abgaben erhöhen.

    \item Im Vergleich zu anderen Städten ähnlicher Größe sind die Übernachtungskosten in Dresden auch mit Bettensteuer sehr günstig. Eine große Auswirkung der Bettensteuer auf die Anzahl der Übernachtungen gibt es daher nicht.
\end{itemize}

Die Piraten fordern jedoch eine Vereinfachung der bürokratischen Abläufe. Ebenfalls muss für die Bettensteuer ein spürbarer Vorteil für die Besuchenden eingeräumt werden, sodass die Abgabe als positiv angesehen wird (z. B. Angebote der DVB oder Rabatt in Museen/Schwimmbädern).


\subsection{Bodenspekulation eindämmen $-$ Grundsteuer C einführen}
Die Grundsteuerreform ermöglicht den Kommunen, eine neue Grundsteuer auf spekulative Baugrundstücke zu erheben (Grundsteuer C, [1]). Die PIRATEN Dresden fordern, die Grundsteuer C in Dresden zum 01.01.2025 einzuführen.\newline

\footnotesize{[1] \tt{https://kommunal.de/grundsteuer-c-bedeutung-wo-eingefuehrt}}


\subsection{Wirtschaftsförderung für Ideenschmieden, Start-ups und Zukunftsindustrien}
Die PIRATEN Dresden sehen die unkommerzielle/kommerzielle Kunst- und Kreativwirtschaft, die Start-up-Szene sowie die (industrielle) Forschung mit ihren Spin-Off's als essenziell für die Zukunft unserer Stadt an. Hier gilt es, die generellen Rahmenbedingungen zu schaffen, damit Kreativität angezogen und gehalten wird. Diese Rahmenbedingungen sollen auch aktiv seitens der Wirtschaftsförderung begleitet und gefördert werden.


\subsection{Städtische Unternehmen}
Dresden hat eine lange Tradition von erfolgreichen kommunalen Betrieben, welche der Stadt wichtige Entscheidungs- und Regelungskompetenzen sichern. Die PIRATEN Dresden fordern, diesen Weg aktiv weiter zu beschreiten $-$ wobei Transparenz und gesellschaftliche Kontrolle der Unternehmen wesentliche Faktoren sind, damit diese nicht der Korruption oder Misswirtschaft verfallen.


\subsection{Dresdner Haushalt}

Position der PIRATEN Dresden zum schuldenfreien Haushalt:

\begin{itemize}
    \item Die PIRATEN Dresden halten an der Schuldenfreiheit des Städtischen Haushalts fest.
    \item Die Schuldenfreiheit darf nicht über Privatisierungen oder Verkäufe kommunalen Eigentums realisiert werden.
    \item Die Finanzierung konsumptiver Ausgaben soll auch nicht über Kredite der städtischen Eigenbetriebe erfolgen.
    \item Auf Nutznießerfinanzierung (DVB) darf nicht verzichtet werden.
\end{itemize}

\section{Tierschutz}

\subsection{Tierschutz bei Pferdefuhrwerken}
Die PIRATEN Dresden fordern, Pferde-Kutschfahrten grundsätzlich zu untersagen. Sollte sich für diese Position keine politische Mehrheit finden, so fordern die PIRATEN Dresden, dass zumindest die Bedingungen für Kutschpferde in der Innenstadt deutlich zu verbessern sind. Die Tiere müssen z. B. über Pausenunterstände auf weichem Boden, Sonnen-/Regenschutz, ausreichend Wasser und Nahrung sowie über eine begrenzte ``Arbeitszeit'' verfügen, um vor tierwohlgefährdender Ausnutzung geschützt zu werden. Kutschbetrieben, die diese Regeln nicht einhalten, müssen Kutschfahrten untersagt werden.


\subsection{Zoo der Zukunft}
Die PIRATEN Dresden bekennen sich zu dem Grundsatz, dass Tiere nicht zum Wohlgefallen des Menschen ausgenutzt werden dürfen. Daher streben wir einen Zoo gänzlich ohne Tiere an.\newline

Ein solcher 'Zoo der Zukunft' hat den Auftrag, die Wichtigkeit des Respekts vor der Natur und den Erhalt der Artenvielfalt sowohl im eigenen (persönlichen) Umfeld als auch im (weit) größeren Rahmen zu fördern. Um dieses Ziel zu erreichen, sind z. B. (interaktive) Lehrpfade und gezielte didaktische Angebote (z. B. 'Schule im Zoo') weit besser geeignet als das bloße Anschauen von eingesperrten Tieren.\newline

Um das Ziel eines tierfreien Zoos zu erreichen, muss der Zoo zunächst sämtliche Nachzucht einstellen. Die dadurch freiwerdenen Kapazitäten können für eine Übergangszeit als Auffangstation für sonst nicht mehr vermittelbare oder auswildbare Tiere (Sanctuary) genutz werden.\newline

Den Auftrag zum Artenschutz (Artenerhalt) soll der Dresdner Zoo über Partnerschaften mit Organisationen in den Herkunftshabitaten bedrohter Tierarten erfüllen. Die Erfolge/Miserfolge dieses 'Artenschutz vor Ort' sowie die Gründe für das Gelingen/Scheitern sind im Zoo erlebbar zu machen.


\subsection{Tierschutz im Zirkus}
Die PIRATEN Dresden sind grundsätzlich gegen Tiere im Zirkus. (Modul B, stimmt das?)


\subsection{Dresden als Stadt für Hunde und deren Haltende}
Die PIRATEN Dresden setzen sich dafür ein, Hundehalter und Hunde im Stadtgebiet Dresden mehr zu unterstützen. Dazu gehört:

\begin{itemize}
    \item Die Schaffung von Hundespielwiesen und anderen Freiflächen, nach Möglichkeit abseits von Wohngebieten.

    \item Das vermehrte zur Verfügung stellen von Hundestationen (kostenlose Kotbeutel und Mülleimer).

    \item Die vermehrte und regelmäßige Leerung von Mülleimern auch abseits der Hauptstraßen.
\end{itemize}


\subsection{Katzen}
Die PIRATEN Dresden lehnen einen generellen Kastrationszwang für Hauskatzen ab. Wir unterstützen aber Informationskampagnen zur Kastration von Katzen, damit Menschen bewusst eine Entscheidung für oder gegen die Kastration eingehen können. Um die Entscheidungsfreiheit zu wahren, soll sich die Stadt bei finanzschwachen Menschen an den Kastrationskosten beteiligen. Außerdem sollen Katzenhäuser und Tierheime aktiv und finanziell unterstützt werden, um aufgenommene Katzen zu versorgen.

\section{Wahlprogramm der Neustadtpiraten zur Kommunalwahl 2024}

Der Ortsverband der Neustadtpiraten hat ein ergänzendes Wahlprogramm aufgestellt. Hierbei wurde der Schwerpunkt auf Neustadtspezifische Themen gesetzt. Das Wahlprogramm der Neustadtpiraten ist vom Kreisverband übernommen worden und gilt somit gesamtstädtisch für die Wahlperiode 2024 bis 2029.




\subsection{Bürgerbeteiligung, Einwohnerversammlungen, Workshops, ... }

\begin{enumerate}

\item Die Neustadtpiraten wirken im Stadtbezirksbeirat und im Stadtrat darauf hin, dass grundsätzlich, im Rahmen einer umfassenden Bürgerbeteiligung, bei wichtigen Themen des Stadtteils Workshops und/oder Einwohnerversammlungen stattfinden müssen. Die Ergebnisse dieser Veranstaltungen sind im weiteren Prozess zu berücksichtigen und die Art dieser Berücksichtigung ist öffentlich zu dokumentieren.

\item Die Neustadtpiraten helfen Bürgerinitiativen, an notwendige Hintergrundinformationen zu gelangen und Rederechte in den Gremien zu ermöglichen.

\item Die Neustadtpiraten verpflichten sich, jeder Petent·in einer Stadtratspetition mit mehr als 3.000 Mitzeichnenden auf Wunsch ein Rederecht im Stadtbezirksbeirat, Fachausschuss oder Stadtrat zu ermöglichen, sofern die Petent·in oder der Petitionsinhalt nicht unserer Unvereinbarkeitserklärung widersprechen.

\end{enumerate}

%
%\subsection{Erklärung zur BRN
%Die Neustadtpiraten bekennen sich zur BRN und wissen um die besondere Bedeutung dieses Stadtteilfestes für %die Äußere Neustadt. Wir legen Wert auf Zusammenarbeit mit den örtlichen Bürger·innen-Bewegungen (z. B. der %Schwafelrunde bzw. deren Nachfolgerinnen) und verfolgen das gemeinsame Bemühen um eine positive Entwicklung %unseres Stadtteilfestes. Aufgrund der steigenden und im Ehrenamt kaum mehr stemmbaren bürokratischen %Belastung ist die Zukunft der BRN noch offen. Wir sind gegen eine fortschreitende Kommerzialisierung und für %mehr Beteiligung der Bewohnenden der Dresdner Neustadt bei der Frage, wie die BRN in Zukunft organisiert %werden soll. Das Fest gehört den Neustädter·innen in ihrer Tradition der unabhängigen Bunten Republik %Neustadt $-$ und deshalb können auch nur sie über ihre Zukunft entscheiden.


\subsection{Position zu Parkhäusern}

Um das Ziel der autofreien Neustadt zu unterstützen, können im Stadtbezirk Parkhäuser an der Peripherie der Neustadt in Kombination mit Quartierbuslinien geschaffen werden. Innerhalb der Wohnquartiere lehnen wir Parkhäuser ab.



\subsection{Verkehrsberuhigung durch Fahrradstraßen}

Im Rahmen der Umsetzung des Ziels ``Autofreie Neustadt'' können Fahrradstraßen eine sinnvolle Teilkomponente sein.


\subsection{Verkehrsberuhigung durch verkehrsberuhigte Bereiche}

Auf dem Weg hin zur Autofreien Neustadt ist die Umwandlung der Timaeus- und Talstraße zu verkehrsberuhigten Bereichen (Spielstraße) ein sinnhafter Schritt.

\subsection{Quartier Jägerpark}
Wir fordern eine Integration des Gebiets Jägerpark in die städtischen Entwicklungskonzepte. Dies beinhaltet u.a. die ÖPNV-Anbindung, den Breitbandanschluss sowie die Weiterentwicklung als Teilgebiet der Neustadt. Die lokalen Akteure vor Ort (z. B. Borea, 19. Grundschule) sind in diesen Prozess mit einzubeziehen.

Insbesondere nach Rückkauf von VONOVIA-Wohnungen in dem Gebiet muss die Stadt eine städtebauliche Qualitätsinitiative starten.


\subsection{Entwicklung Altes Postgelände}
Wir sprechen uns dafür aus, das Alte Postgelände in seiner Gesamtheit als städtebauliche Erweiterung der Äußeren Neustadt zu begreifen und einer öffentlichen Nutzung zuzuführen. Denkbar wäre z. B. eine Schule in Kombination mit einem Jugend- und Freizeitzentrum. Das jetzige Bürogebäude könnte als ``Inkubator'' oder als ``Kreativhaus'' genutzt werden.


\subsection{``Putzi''-Gelände}
Wir wollen, dass der zurzeit leer stehende Platz auf dem ``Putzi''-Gelände für die Öffentlichkeit zugänglich gemacht wird. Die leer stehenden Villen eignen sich zur Zwischennutzung als Wächterhäuser. Langfristig sind kulturelle Nutzungen bzw. die Einrichtung eines selbstverwalteten Jugendzentrums denkbar. Die Betonfläche vor dem Fabrikgebäude eignet sich als Kleinpark mit Großgrün sowie Spiel- und Sportplatz.


\subsection{Westerweiterung Alaunplatz}
Wir fordern eine zügige Vollendung der Westerweiterung des Alaunplatzes über das Gebiet des ehemaligen ``Russensportplatz'' hinaus bis an die Tannenstraße und in Richtung Königsbrücker Straße. Dies wird endlich einen barrierefreien nördlichen Zugang ermöglichen. Bei der Erweiterung ist die Schaffung von Sport- und Freizeitangeboten in Form von öffentlichen und frei zu benutzenden Sportstätten vorzusehen (zum Beispiel Bolzplatz, Basketballplatz, Beachvolleyballplatz, Half-Pipe/Pump-Track, BMX-Hügel, Tischtennisplatten, Schach-/Backgammontischen, Feuerstelle, öffentlicher Pizzabackofen, evtl. Hundewiese). Bei Auswahl der Nutzungen und Gestaltung der Anlagen ist eine fundierte Öffentlichkeitsbeteiligung durchzuführen.


\subsection{Grillen im Alaunpark}
Die Neustadtpiraten sprechen sich gegen ein generelles Grillverbot im Alaunpark aus.


\subsection{S-Bahn-Haltepunkt Bischofsplatz und Umgestaltung}
Wir unterstützen die Umgestaltung des Bischofplatzes, welche jetzt nach dem Bau des S-Bahnhaltepunktes Bischofsplatz dringender denn je erfolgen muss. Hierbei ist auf eine Steigerung der Aufenthaltsqualität innerhalb einer Gesamtkonzeption (Geschäfte, Cafés, Dienstleistung, Carsharing, Fahrradabstellanlagen) Wert zu legen. Der Fokus der baulichen Umgestaltung muss auf der Benutzbarkeit des Platzes als öffentlicher Raum liegen. Hierzu gehören neben einer sicheren Überquerbarkeit des Bischofsplatzes für Fußgehende und Radfahrende auch eine autofreie Nutzung des Straßenraums der Conrad- und Eschenstraße.\newline

Die städtebaulichen Potenziale einer Integration der Fläche der Bahn AG (zwischen Eschenstraße, Bischofsplatz und Bahndamm) sind bei den Planungen auszuschöpfen.




\subsection{Quartierbus Neustadt}
Wir forden die Einrichtung von Quartierbuslinien, die die Stadtteile Äußere Neustadt, Hechtviertel, Preußisches Viertel und das Gebiet um den Jägerpark verbinden. Dies würde die Bereitschaft, eine ÖPNV-Dauerkarte zu erstehen, erhöhen, den lokalen Einzelhandel stärken, die betreffenden Gebiete vom Kfz-Verkehr entlasten und zu einer besseren Anbindung der Stadtteile an den (überregionalen) ÖPNV führen.


\subsection{Behindertenparkplätze}
Parkplätze für Schwerbehinderte müssen innerhalb der Neustadt beibehalten und ausgeweitet werden, bis das Gesamtkonzept ``Autofreie Neustadt'' komplett barrierefrei umgesetzt wurde.


%
%Autofreie Äußere Neustadt
%Die Neustadtpiraten streben das Ziel der \u201eAutofreien Äußeren Neustadt\u201c an. Ähnlich der Superblocks %in Barcelona wird der heute durch (parkende) Autos ineffizient genutzte Stadtraum dadurch öffentlich für %alle nutzbar \u2013 \u201eReclaim the Street\u201c wird Wirklichkeit.
%
%Als erste Schritte in diese Richtung fordern wir:
%
%* Parkverbot in Louisen-, Alaun-, Kamenzer, Rothenburger und Görlitzer Straße sowie am Martin-Luther-Platz.
%* Einführung von Stellplätzen für Lastenräder/E-Bikes (inkl. Ladesäulen).
%* Nutzung des freigewordenen Raums zum Flanieren, für Radbügel, Stadtgrün und für den Einzelhandel sowie die %Gastronomie.
%* Freigabe des Straßenraums für den Fußverkehr.
%* Verhinderung der Durchquerbarkeit der Äußeren Neustadt für den KFZ-Verkehr.
%* Massive Erhöhung der Parkgebühren für die noch verbleibenden Parkplätze.
%* Einführung eines Quartierbusses zur Anbindung der Parkhäuser in der Peripherie.
%* Ausweisung der Parkplätze an den Außengrenzen (z. B. Königsbrücker, Bischofsweg, Bautzner) für Car-Sharing %Angebote.
%* Bike-Sharing-Angebot im gesamten Viertel.



\subsection{Autonomie der Neustadt von Sachsen und Deutschland (Bunte Republik Neustadt)}

Wir fordern die Erneuerung der Bunten Republik Neustadt, welche bereits 1990 proklamiert wurde, als autonome Region. Eichhörnchen im Hanfgewand sollen unsere basisdemokratischen König·innen sein. Außerdem erachten wir eigene Briefmarken, eine Armee aus Flausch, freies, uneingeschränktes Internet und überall Wireless-LAN-Kabel als notwendig. Auch sollen Trinkbrunnen aus denen das Bier sprudelt unsere Straßenecken zieren die unweit von Hanfgewächsen zur Straßenbegrünung zu finden sind. Im Sinne der Währungsunion wird der Euro [alternativ: Liebe] als Ablösung der Neustadtmark akzeptiert. Die Bunte Republik hat die Steuerhoheit und entscheidet selbst, wie die Steuern im Inneren und nach außen verteilt werden. So stehen ein Bedingungsloses Grundeinkommen und die explizite Förderung des Spätverkaufs im Vordergrund. Politisches Asyl wird allen Menschen gewährt, die aktiv gegen Rassismus, Queerfeindlichkeit, Sexismus sowie Menschen- und Bürgerrechtsverletzungen arbeiten oder diese aufdecken (Whistleblower) und deswegen verfolgt werden.










%\addcontentsline{toc}{section}{Präambel}
%\addcontentsline{toc}{section}{Netzpolitik}
%\addcontentsline{toc}{section}{Freiheit und Selbstbestimmung}
%\addcontentsline{toc}{section}{Gleichberechtigung und Gleichstellung}
%\addcontentsline{toc}{section}{Umwelt und Klima}
%\addcontentsline{toc}{section}{Jugend und Bildung}
%\addcontentsline{toc}{section}{Soziales und Wohnen}
%\addcontentsline{toc}{section}{Asyl und Migration}
%\addcontentsline{toc}{section}{Kultur}
%\addcontentsline{toc}{section}{Sport}
%\addcontentsline{toc}{section}{Verkehr}
%\addcontentsline{toc}{section}{Stadtentwicklung}
%\addcontentsline{toc}{section}{Wirtschaft/Finanzen}
%\addcontentsline{toc}{section}{Tierschutz}




% Merken

%\begin{enumerate}
%    \item
%    \item
%    \item
%    \item
%    \item
%    \item
%\end{enumerate}

%\vspace*{0cm}\hspace{2cm}\includegraphics[width=8cm]{Pictures/schild_afmsw23.jpg}%\par\vspace{6cm}
%\begin{figure}[hhh]
%	%\centering
%	\vspace*{0cm}\hspace{4cm}\includegraphics[width=6cm]{Pictures/schild_afmsw23.jpg}%\par\vspace{6cm}
%	\caption{Quelle: ADFC Dresden, {@}rw\_d\_wr}
%	\label{img:grafik-dummy}
%\end{figure}

%\begin{center}
%\begin{tabular}{ | m{9cm} | m{1cm}| m{1cm} | }
%\hline
%\textbf{Abschnitt} & Datum  & Kfz/Tag \\
%\hline
%\hline
%cell1 dummy text dummy text dummy text& cell2 & cell3 \\
%\hline
%cell1 dummy text dummy text dummy text & cell5 & cell6 \\
%\hline
%cell7 & cell8 & cell9 \\
%\hline
%\end{tabular}
%\end{center}



%[1]  {\footnotesize {\tt https://twitter.com/ADFC\_Dresden/status/1293509334987681793}


% das ist wohl jetzt das Ende des Dokumentes
\end{document}
